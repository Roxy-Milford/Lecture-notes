\chapter{Геометричний ймовірнісний експеримент}

\begin{problem}
    Нехай  --- деяка область скінченної міри (довжина в , площа в ,
    об'єм в , ). В цій області ми навмання вибираємо деяку точку  .
    Питання: чому дорівнює ймовірність того, що вибрана точка буде
    належати деякій області ? 
\end{problem}

Припущення:

\begin{enumerate}
    \item Вибрати можна довільну точку .
    \item Ймовірність потрапляння точки в область
    має бути пропорційно мірі  (довжині, площі, об'єму,  )

    \item Ймовірність не має залежати від розміщення і форми
    області .
\end{enumerate}

При виконанні припущень 1) -- 3) покладають



де  --- це міра у відповідному просторі.

\subsection*{Приклади:}

1) Задача про зустріч.

Двоє осіб домовилися про зустріч між 12:00 і 13:00. Кожний
вибирає час приходу навмання, чакає не більше 15хв. і йде.
Яка ймовірність, того, що ці дві особи зустрінуться?

--- час прибуття  -щї особи.



2) Розглянемо квадратне рівняння  , де , вибираються
навмання в , незалежно одне від одного.

Корені рівняння виявляються дійсними

необхідна і достатня умова того, що корені дійсні.

\begin{problem}[парадока Бертрана]    
    В крузі одиничного радіусф навмяння вибирається хорда.
    Яка ймовірність того, що її довжина буде більшою за 
    сторону правильного трикутника, вписаного в це коло.
\end{problem}
\begin{solution}
    Варіант 1: Закріпимо один кінець хорди. Тоді другий
    кінець --- це довільна точка на колі:

    Варіант 2: Фіксуємо напрямок хорди. Проведемо діаметр,
    перпендикілярний до цього напрямку. Кожну хорду можна
    ототожнити з точкою цього діаметра

    --- точка відрізка.

    Варіант 3: Через кожну точку круга проходить єдина
    хорда, для якої ця точка круга є серединою. Тому хорду
    можна ототожнити з її серединою.  --- середина хорди

    --- множина точок круга, вписаного в правильний трикутник.

\end{solution}


\begin{example}
    (Задача Бюффона) Голка довжини $2$ навмання кидається на
    площину, розграфлену паралельними прямими на відстані
    $2$ одна від одной. Знайти ймовірність того, що голка
    перетне якусь з прямих.
\end{example}

\begin{solution}    
    Положення голки на площині опишемо парою (а,а), де
    х- відстань від центра голки до ближчої з прямих;
    фі-кут, утворений голкою з вказаною прямою
    
    Обчислення числа пі методом Монте-Карло: т разів голку
    кинуто на площину. Нехай К разів голка перетнула пряму.
    Число к/н має бути близьким (при дуже великому п) до
    Отже  --- апроксимація для д .
\end{solution}


\chapter{Симетричне випадкове блукания. принцип відбиття (віддзернаяемия)}

Нехай п- фіксований час спостережения
Л- множина всіх можливих траєкторій частинки за час н

на  -тому кроці, крок був вгору

на  -тому кроці, крок був вниз

Покладемо: .

Траекторія випадкового блукання називається шляхом з початку
кординат

Нехай  -кількість шляхів з точки

Умови досяжності:  і  і  мають мати однакову парність

--- кількість кроків вгору

--- кількість кроків вниз

--- кількість кроків (всього)



Множина шляхів, які ведуть з  в

\begin{remark}
    при 
\end{remark}

\begin{lemma}[Принцип відбиття]
    Нехай  і --- точки з цілочисельними кординатами  і  відповідно,
    причому  , .
\end{lemma}

Нехай  --- точка з кординатами  ---симетрична точці  відносно
осі абсис.

Тоді кількість шляхів з  в , які дотикаються або перетинають вісь 
абсис, дорівнює кількості всіх шляхів і  в  .


Кожному шляху з  в  , фкий дотикається або перетинає вісь абсис,
поставимо у відповідність шлях з  в  наступним чином:
якщо шлях з  в  попадпє на вісь х вперше в точці С, то ділянку
шляху з  в  будуємо як відображення з  в  без змін ділянку з 
в

Така відповідність є взаємооднозначною, що й доводить лему.

Приклади:

а) знайдемо кількість додатніх шляхів х  в  (всі шляхи над
віссю абсис)

б)Кількість невідємних шляхів з  в  
--- кількість невідємних шляхів, які ведуть з  в  .

