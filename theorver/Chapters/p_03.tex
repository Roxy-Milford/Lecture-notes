\chapter{Геометричний ймовірнісний експеримент}

\begin{problem}
    Нехай $\Omega$ --- деяка область скінченної міри (довжина в $\mathbb{R}$,
    площа в $\mathbb{R}^2$, об'єм в $\mathbb{R}^m$, $m \geqslant 3$). В цій
    області ми навмання вибираємо деяку точку $\omega in \Omega$.
    Чому дорівнює ймовірність того, що вибрана точка буде
    належати деякій області $A \subseteq \Omega$? 
\end{problem}

Припущення:

\begin{enumerate}
    \item Вибрати можна довільну точку $\Omega$.
    \item Ймовірність потрапляння точки в область $A$
    має бути пропорційною мірі $A$ (довжині, площі, об'єму $A$)

    \item Ймовірність не має залежати від розміщення і форми
    області $A$.
\end{enumerate}

При виконанні припущень 1) -- 3) покладають

$$P(A) = \dfrac{\mu(A)}{\mu(\Omega)}$$

де $\mu(X)$ --- це міра у відповідному просторі.

\subsection*{Приклади:}

1) Задача про зустріч.

Двоє осіб домовилися про зустріч між 12:00 і 13:00. Кожний
вибирає час приходу навмання, чакає не більше 15хв. і йде.
Яка ймовірність, того, що ці дві особи зустрінуться?

$t_i$ --- час прибуття  $i$-тої особи.

$\Omega = \{(t_1, t_2): 0 \leqslant t_1 \leqslant 1, 0 \leqslant t_2 \leqslant 1\}$

$A = \{(t_1, t_2) \in \Omega: |t_1-t_2| \leqslant \frac{1}{4}\}$

\begin{center}
    Beautifull image
\end{center}

$\Omega = [0,1]^2$

$P(A) = \dfrac{S_A}{S_{\Omega}}
= \dfrac{1-2\frac{1}{2}(\frac{3}{4})^2}{1}
= 1 - \dfrac{9}{16}
= \dfrac{7}{16}$

2) Розглянемо квадратне рівняння $X^2 + ax + b = 0$, де $a$, $b$ вибираються
навмання в $(0, 1)$, незалежно одне від одного.

$A = \{$Корені рівняння виявляються дійсними$\}$

$\Omega = \{ (a, b) \in \mathbb{R}^2: 0 < a < 1, 0 < b < 1\}$

$D = a^2 - 4b \geqslant 0$ --- необхідна і достатня умова того, що корені дійсні.

$A = \{(a, b) \in \Omega: b \leqslant \frac{a^2}{4}\}$

\begin{center}
    Beautifull image
\end{center}

$\Omega = [0, 1]^2$

$P(A) = \dfrac{S_A}{S_{\Omega}}
= \dfrac{\int\limits_0^1 \dfrac{a^2}{4}da}{1}
= \dfrac{1}{12}$

\begin{problem}[парадокс Бертрана]    
    В крузі одиничного радіусф навмяння вибирається хорда.
    Яка ймовірність того, що її довжина буде більшою за 
    сторону правильного трикутника, вписаного в це коло.
\end{problem}
\begin{solution}
    Варіант 1: Закріпимо один кінець хорди. Тоді другий
    кінець --- це довільна точка на колі:

    \begin{center}
        Beautifull image
    \end{center}

    $\Omega = [0, 2\pi]$

    $A = [\dfrac{2\pi}{3}, \dfrac{4\pi}{3}]$

    \begin{center}
        Beautifull image
    \end{center}

    $P(A = \dfrac{\frac{4\pi}{3}-\frac{2\pi}{3}}{2\pi} =\dfrac{1}{3}$

    Варіант 2:
    
    \begin{center}
        Beautifull image
    \end{center}
    
    Фіксуємо напрямок хорди. Проведемо діаметр,
    перпендикілярний до цього напрямку. Кожну хорду можна
    ототожнити з точкою цього діаметра

    $\Omega = [0, 2]$ --- точка відрізка $[0, 2]$.

    $A = [\dfrac{1}{2}; \dfrac{3}{2}]$

    $P(A) = \dfrac{\frac{3}{2}-\frac{1}{2}}{2} = \dfrac{1}{2}$.

    Варіант 3:
    
    \begin{center}
        Beautifull image
    \end{center}

    Через кожну точку круга проходить єдина
    хорда, для якої ця точка круга є серединою. Тому хорду
    можна ототожнити з її серединою. $(x, y)$ --- середина хорди

    $\Omega = \{(x, y): x^2 + y^2 < 1\}$

    $A$ --- множина точок круга, вписаного в правильний трикутник.

    $A = \{(x, y) \in \Omega: x^2 + y^2 \leqslant \dfrac{1}{4}\}$

    $P(A) = \dfrac{S_A}{S_{\Omega}} = \dfrac{\frac{1}{4}\pi}{\pi} = \dfrac{1}{4}.$
\end{solution}


\begin{example}
    (Задача Бюффона) Голка довжини $2$ навмання кидається на
    площину, розграфлену паралельними прямими на відстані
    $2$ одна від одной. Знайти ймовірність того, що голка
    перетне якусь з прямих.
\end{example}

\begin{solution}
    \begin{center}
        Beautifull image
    \end{center}
    
    Положення голки на площині опишемо парою $(x, \varphi)$, де
    $x$ --- відстань від центра голки до ближчої з прямих;
    $\varphi$-кут, утворений голкою з вказаною прямою

    \begin{center}
        Two beautifull images
    \end{center}

    $\Omega = \{(x, \varphi) \in \mathbb{R}^2: 0 \leqslant x \leqslant 1,
        0 \leqslant \varphi \leqslant \pi \}$

    $A = \{(x, \varphi) \in \Omega: x \leqslant \sin(\varphi)\}$

    $P(A) = \dfrac{S_A}{S_{\Omega}}
    = \dfrac{\int\limits_0^{\pi} \sin(\varphi) d \varphi}{1\pi}
    = \dfrac{2}{\pi}$

    \begin{center}
        Beautifull image
    \end{center}
    
    Обчислення числа $\pi$ методом Монте-Карло: $n$ разів голку
    кинуто на площину. Нехай $k$ разів голка перетнула пряму.
    Число $\dfrac{k}{n}$ має бути близьким (при дуже великому $n$)
    до $\dfrac{2}{\pi}$. Отже $\pi \approx \dfrac{2n}{k}$
    --- апроксимація для $\pi$.
\end{solution}

\chapter{Симетричне випадкове блукания. принцип відбиття (віддзеркалення)}

\begin{center}
    Beautifull image
\end{center}

Нехай $n$ --- фіксований час спостережения.
$\Omega$ --- множина всіх можливих траєкторій
частинки за час $n$.

$\Omega = \{(\varepsilon_1, \varepsilon_2, ..., \varepsilon_n),
    \varepsilon_i \in \{-1, 1\},
    i = \overline{1, n}\}$

$\varepsilon_i = 1$ на  $i$-тому кроці, крок був вгору

$\varepsilon_i = -1$ на  $i$-тому кроці, крок був вниз

$|\Omega| = 2^n$

Покладемо: $\forall \omega \in \Omega$.

Траекторія випадкового блукання називається шляхом з початку
кординат

Нехай  -кількість шляхів з точки

Умови досяжності:  і  і  мають мати однакову парність

--- кількість кроків вгору

--- кількість кроків вниз

--- кількість кроків (всього)



Множина шляхів, які ведуть з  в

\begin{remark}
    при 
\end{remark}

\begin{lemma}[Принцип відбиття]
    Нехай  і --- точки з цілочисельними кординатами  і  відповідно,
    причому  , .
\end{lemma}

Нехай  --- точка з кординатами  ---симетрична точці  відносно
осі абсис.

Тоді кількість шляхів з  в , які дотикаються або перетинають вісь 
абсис, дорівнює кількості всіх шляхів і  в  .


Кожному шляху з  в  , фкий дотикається або перетинає вісь абсис,
поставимо у відповідність шлях з  в  наступним чином:
якщо шлях з  в  попадпє на вісь х вперше в точці С, то ділянку
шляху з  в  будуємо як відображення з  в  без змін ділянку з 
в

Така відповідність є взаємооднозначною, що й доводить лему.

Приклади:

а) знайдемо кількість додатніх шляхів х  в  (всі шляхи над
віссю абсис)

б)Кількість невідємних шляхів з  в  
--- кількість невідємних шляхів, які ведуть з  в  .

