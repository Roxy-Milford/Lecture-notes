\chapter{Умовні ймовірності. Незалежність подій. Формула повної ймовірності та формула Байса}

Умовна ймовірність події  в припущенні, що відбулася деяка
подія  визначається так:



де вважається, що  .

Приклади

1) Підкидаємо два гральних кубика.

На першому кубику випало 6 очок

сума всіх очок дорівнює 11



2) В сімї є двоє дітей. Відомо, що одне з дітей --- це
хлопчик. Яка ймовірність того, що в цій сімї є
дівчинка?


Вважаємо, що всі елементарні події є рівноможливими.

В сімї є дівчинка

В сімї є хлопчик


Таким чином:

Твердження Властивості умовної ймовірності


зауваження: що події  і  є несумісними 


Тому

Отже


2) Запишемоправу частину


Випадкові події  і  --- це незалежні події, якщо 


Приклади

Монета підкидається двіччі

При першому підкиданні випав герб

При другому підкиданні випав герб

В обох підкиданнях випав герб


Таким чином

Отже події  є незалежними

Події   --- це незалежні в сукупності події,якщо

для довільного набору випадкових подій


Приклад Піраміда Бернштейна

Правильна чотирикутна піраміда пофарбована наступним чином:
є червона, синя , зелена грані і одна грань, що пофарбована
у всі три кольори. При підкиданні кожна грань випадає з
ймовірністю

Випала грань, яка має червоний колір

Випала грань, яка має зелений колір

Випала грань, яка має синій колір

Отже і  і  і  є незалежними

розглянемо  :

Отже події ,  і  не є незалежними в сукупності

Твердження:
1) Якщо  і  незалежні і  то
2) Якщо  і  є несумісними і  , то  і  не є незалежними.


1) Якщо  і  незалежні, то

тоді

2) Якщо  і  несумісні, то  . Тому  . В той же час .
Значить  , а отже  і  не є незалежними.


Набір подій  --- це повна група подій, якщо:
1)  при   і   є несумісними
2) 
3) для довільного .

Кажуть, що  визначають розббиття  .

Теорема формула повної ймовірності

Нехай  --- повна група подій на і нехай  . тоді


Оскільки  є попарно неперетинними, то 
при  . Тому

Теорема  формула Байеса

Нехай  --- це повна група подій на і  . тоді

Оскільки   ,  то .

Ймовірності ,  --- апріорні ймовірності,  --- апосторіорні
ймовірності гіпотез.

Приклад:

1) маємо 4 кулі: 2 білих, 2 чорних кулі, розкладають
довільним чином по двом вазам.

Правитель навмання вибирає вазу, з якої навмання виймає кулю.
Яка ймовірність того, що куля виявиться білою.

вийняли білу кулю

вибрали 1-шу вазу  ,  вибрали 2-гу вазу

1-й спосіб

2-й спосіб

3-й спосіб

2) Підкидається гральний кубик, а потім монета підкидається
стільки разів, скільки очок випало на кубику. Відомо, що
при підкиданні монети випали всі герби. Яка ймовірність
того, що на кубику випало 5 очок?

При підкиданні монети випали всі герби

На кубику випало  очок

Скористаємося формулою Байеса:

Обчислюємо:

при  підкиданнях монети випали всі герби

Отже

3) Повернення в 0 випадкового симетричного блукання

Частинка стартує з 0, в дискретні моменти часу  робить 
крок вгору, вниз з ймовірністю  . Нехай  --- мовірність
того, що частинка коли-небуть повернеться у вихвдну точку.

Запишемо формулу повної ймовірності для  , зробивши
гіпотези про результат -го кроку.

введемо позначення

частина потрапить в 0, стартувавши з точки 

тоді
1-ий крок вгору   1-ий крок вниз

Запишемо формулу повної ймовірності для 

Характеристичне рівнянняарифметричної прогресії, а отже

Оскільки  , то . Отже

Підставимо  в рівняння для :


Таким чином  і тоді

Висновок: з ймовірністю 1 частинка, стартувала з 0, повернеться коли небуть
в 0.


4) Задача про розорення гравця.

Гравець починає гру в казино, маючи 1 грн.

За одну гру гравець виграє одну гривню з ймовірністю 
або програє 1 грн з ймовірністю .

Казано має  гривень. Яка ймовірність розорення гравця?

Позначимо

--- ймовірніст розорення гравця, який має  гривень.

Зрозуміло, що 

Запишемо для  формулу повної ймовірності

Отримали

Підсумуємо обидві частини від 0 довільним

Зліва

Справа

Прирівняємо

Звідси

--- ймовірність розорення гравця

--- ймовірність розорення казино

Припустимо, що .Тоді 

а) якщо  . Тоді

б) . Тоді

Для того, щоб ймовірність розорення гравця була меншою
за ймовірність розорення казино потрібно 

