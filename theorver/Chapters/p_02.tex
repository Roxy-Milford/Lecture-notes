Розглянемо подію 

$A_1 = \{$ Лист з номером 1 прийшов за призначенням $\} =$

$= \left\{\begin{pmatrix}
    1 & 2 & 3 & ... & n\\
    1 & i_2 & i_3 & ... & i_n \\
\end{pmatrix} \in \Omega \right\}$, $|A_1| = (n-1)!$

Зрозуміло, що $|A_2| = ... = |A_n| = (n-1)!$

$A_1 \cap A_2 = \{$ Листи з номерами 1 і 2 прийшли за призначенням $\} =$

$= \left\{\begin{pmatrix}
    1 & 2 & 3 & ... & n\\
    1 & 2 & i_3 & ... & i_n \\
\end{pmatrix} \in \Omega \right\}$, $|A_1 \cap A_2| = (n-2)!$

і так далі...

$|A_1 \cap A_3 \cap ... \cap A_k| = (n-k)!$

$P\left(\bigcup\limits_{i=1}^n A_i\right)
= \sum\limits_{i=1}^n \dfrac{(n-1)!}{n!}
    - \sum\limits_{i<j} \dfrac{(n-2)!}{n!}
    + \sum\limits_{i<j<k} \dfrac{(n-3)!}{n!}
    + ...
    + (-1)^{n-1} \dfrac{1}{n!} =$

$= n \dfrac{(n-1)!}{n!}
    - C_n^2 \dfrac{(n-2)!}{n!}
    + C_n^3 \dfrac{(n-3)!}{n!}
    + ... 
    + (-1)^{n-1} \dfrac{1}{n!} =$

$= \dfrac{(n-1)!}{n!}
    - \dfrac{n!}{2!(n-2)!} \dfrac{(n-2)!}{n!}
    + \dfrac{n!}{23(n-3)!} \dfrac{(n-3)!}{n!}
    + ... 
    + (-1)^{n-1} \dfrac{1}{n!} =$

$= 1
    - \dfrac{1}{2!}
    + \dfrac{1}{3!}
    + ... 
    + (-1)^{n-1} \dfrac{1}{n!}$

Якщо $n \rightarrow \infty$, то
 
$$P\left(\bigcup\limits_{i=1}^n A_i\right) \ \xrightarrow[n \rightarrow \infty]{} e ^{-1}$$

\chapter{Деякі класичні моделі і розподіли.}

\section{Біноміальний розподіл}

Припустимо, що деякий ймовірнісний експеримент повторюється $n$ разів, в
кожному з яких може відбутись або дуяка подія $A$ --- $"$успіх$"$, або подія
$\overline{A}$ --- $"$невдача$"$. Простір всіх можливих результатів можна
описати так:

$\Omega = \{\omega: \omega = (\varepsilon_1, ..., \varepsilon_n), \varepsilon_i \in \{0, 1\}\}$, 
де $\varepsilon_i = 0$ , якщо в $i$-тому експерименті сталася невдача, $\varepsilon_i = 1$, якщо
в $i$-тому експерименті стався успіх.

Припишемо кожній елементарній події $\omega = (\varepsilon_1, ..., \varepsilon_n)$ ймовірність

$$P(\omega) = p^{\sum\limits_{i=1}^n \varepsilon_i} (1-p)^{n - \sum\limits_{i=1}^n \varepsilon_i},$$

де $p \in (0, 1)$ - деяке число.

Переконаймося, що означення коректне, тобто, що $\sum\limits_{\omega \in \Omega} P(\omega) = 1$, маємо

$$\sum\limits_{\omega \in \Omega} P(\omega)
= \sum\limits_{(\varepsilon_1, ..., \varepsilon_n) \in \{0, 1\}}
    p^{\sum\limits_{i=1}^n \varepsilon_i}
    (1-p)^{n - \sum\limits_{i=1}^n \varepsilon_i} =$$

$$= \sum\limits_{k=0}^n \sum\limits_{(\varepsilon_1, ..., \varepsilon_n) : \sum \varepsilon_i = k}
    p^k (1-p)^{n-k} = $$

$$= \sum\limits_{k=0}^n p^k (1-p)^{n-k} \sum\limits_{(\varepsilon_1, ..., \varepsilon_n) : \sum \varepsilon_i = k} 1 =$$

$$= \sum\limits_{k=0}^n C_n^kp^k(1-p)^{n-k} =$$

$$= (p+(1-p))^n = 1.$$

При $n=1$ отримаємо:

$$\Omega = \{0, 1\}$$

$$p(0) = 1-p$$

$$p(1) = p$$

$p$ --- це ймовірність $"$успіху$"$ в одному випробуванні.

Розглянемо події

$A_n(k) = \{(\varepsilon_1, ..., \varepsilon_n) \in \Omega: \sum\limits_{i=1}^n \varepsilon_i = k\}
= \{$В $n$ випробуваннях сталося рівно $k$ успіхів$\}$, $k = \overline{0,n}$.

Тоді $P(A_n(k)) = \sum\limits_{\omega \in A_n(k)} p(\omega) = C_n^k p^k (1-k)^{n-k}$,
$k = \overline{0, n}$.

\begin{definition}
    Набір імовірностей $\{P_n(k) = C_n^kp^k(1-p)^{n-k}, k = \overline{0, n}\}$ --- цей
    \textbf{біноміальний розподіл} з параметрами $n$ (кількість випробувань), і  $p$
    (ймовірність успіху в одному випробуванні); $Bin(n,p)$.
\end{definition}

\begin{example}(Випадкове блукання)

    Деяка частинка виходить з нуля і через одиницю часу, робить або крок вгору,
    або вниз. За $n$ кроків, де $n$ фіксоване, частинка може переміститись що
    найбільще на $n$ кроків вгору або вниз.
\end{example}

\begin{center}
    Beautifull image.
\end{center}

Прстір елементарних подій:

$\Omega = \{(\varepsilon_1, ..., \varepsilon_n), \varepsilon_i = \pm 1\}$, де
$\varepsilon_i = -1$, якщо на $i$-тому кроці частинка зробила крок вниз,
$\varepsilon_i = 1$, якщо на $i$-тому кроці частинка зробила крок вгору.

Покладемо

$$P(\omega) = p^{\nu(\omega)} (1-p)^{n - \nu(\omega)},$$

де $\nu(\omega)$ --- це кількість одиничок в $\omega = (\varepsilon_1, ..., \varepsilon_n)$

$$\sum\limits_{i=1}^n \varepsilon_i = \mu(\omega) - (n - \mu(\omega)) = 2\mu(\omega) - n \Rightarrow$$

$$\mu(\omega) = \dfrac{\sum\limits_{i=1}^n \varepsilon_i + n}{2}$$

Оскільки $\sum\limits_{\omega \in \Omega} p(\omega) = 1$, то простір $\Omega$ разом з
розподілом ймовірностей $p(\omega)$ визначає деяку ймовірнісну модель руху частинки
за $n$ кроків.

Розглянемо події

$A_n(k) = \{(\varepsilon_1, ..., \varepsilon_n) \in \Omega: \mu(\omega) - (n - \mu(\omega)) = k\} = \{$
за $n$ кроків частинка опиниться в точці з кординатою $k\}$.

\begin{center}
    Beautifull image.
\end{center}

Червоний прямокутник --- це область де $"$живуть$"$ шляхи з $(0,0)$ в $(n,k)$

$P(A_n(k)) = C_n^{\dfrac{n+k}{2}} p^{\dfrac{n+k}{2}} (1-p)^{\dfrac{n-k}{2}}$

Умова досяжності: $n$ і $k$ мають однакову парність.

\section{Мультиполіноміальний розподіл}

Проведемо $n$ випробувань, в кожному з яких може спостерігатись один з $r$ несумісних 
результатів. Опишемо $\Omega:$

$\Omega = \{(\varepsilon_1, ..., \varepsilon_n), \varepsilon_i = \overline{1,r}\}$,
де $\varepsilon_i = j$ в $j$-му випроуванні спостерігали результат номер $j$,
$j = \overline{1,r}$, $i = \overline{1, n}$.

Нехай:

$\nu_1(\omega)$ --- це кількість одиничок в $\omega = (\varepsilon_1, ..., \varepsilon_n)$

$\nu_2(\omega)$ --- це кількість двійок в $\omega = (\varepsilon_1, ..., \varepsilon_n)$

$\nu_r(\omega)$ --- це кількість координат які дорівнюють $r$ в $\omega = (\varepsilon_1, ..., \varepsilon_n)$

Покладемо: 

$P(\omega) = p_1^{\nu_1(\omega)}p_2^{\nu_2(\omega)}...p_r^{\nu_r(\omega)}$, де $p_i \geqslant 0$,
$\sum\limits_{i=1}^r p_i = 1$.

Покажемо, що $\sum\limits_{\omega \in \Omega} P(\omega) = 1$. Маємо: 

$$\sum\limits_{\omega \in \Omega} P(\omega)
= \sum\limits_{(\varepsilon_1, ..., \varepsilon_n) \in \Omega}
    p_1^{\nu_1(\omega)}p_2^{\nu_2(\omega)}...p_r^{\nu_r(\omega)}
= \sum\limits_{\begin{matrix}
        n_1 \geqslant 0, ..., n_r \geqslant 0\\
        n_1 + ... + n_r = n
    \end{matrix}} 
    p_1^{n_1} p_2^{n_2}...p_r^{n_r} = $$

$$= \sum\limits_{n_1 + ... + n_r = n} 
    C_n^{n_1} C_{n-n_1}^{n_2} ... C_{n-(n_1 + ... + n_r)}^{n_r} p_1^{n_1} p_2^{n_2}...p_r^{n_r}
= (p_1 + ... + p_r)^n = 1.$$

Розглянемо події

$A_n(K_1, ..., K_r) = \{\omega \in \Omega: \nu_1(\omega) = K_1, ..., \nu_r(\omega) = K_r\} = \{$
в $n$ випробуваннях результат $N_1$ відбувся $K_1$ раз, результат $N_2$ відбувся $K_2$ разів, ...,
результат $N_r$ відбувся $K_r$ разів $\}$, $k_1 + ... + K_r = n$.

Тоді

$P(A_n(K_1, ..., K_r)) = \dfrac{n!}{K_1!...K_r!} p_1^{K_1} ... p_r^{K_r}.$

\begin{definition}
    Набір ймовірностей $\{\dfrac{n!}{K_1!...K_r!} p_1^{K_1} ... p_r^{K_r}, K_i \geqslant 0, K_1 + ... + K_r = n\}$
    --- це \textbf{мультиполіноміальний (поліноміальний) розподіл} з параметрами $n$ і $p_1, ..., p_r$, де
    $\sum\limits_{i=1}^r p_i = 1.$
\end{definition}

\begin{remark}
    І біноміальний і поліноміальний розподіли пов'язані з вибором без повторень.
\end{remark}

\section{Гіпергеометричний роподіл}

Нехай в урні міститься $N$ куль, занумерованих від $1$ до $N$. З них $M$ є білими,
$N-M$ чорними. Припустимо, що проводиться вибір без повернення $n$ куль,
$n < N$. Простір елементирних подій: 

$$\Omega = \{\omega: \omega = (\varepsilon_1, ..., \varepsilon_n),
    \varepsilon_i = \overline{1, N},
    \varepsilon_1 \neq \varepsilon_2 \neq ... \neq \varepsilon_n\}$$

$$|\Omega| = N(N-1)...(N-n+1) = A_N^n$$

Покладемо: $\forall \omega \in \Omega: P(\omega) = \dfrac{1}{|\Omega|} = \dfrac{1}{A_N^n}$

Розглянемо події:

$A_n(k) = \{$Серед  вийнятих куль рівно $k$ виявились білими$\}$

$|A_n(k)| = \dfrac{n!}{k!(n-k)!} A_M^k A_{N-M}^{n-k}$

і тоді 

$P(A_n(k)) = \dfrac{|A_n(k)|}{|\Omega|}
= \dfrac{n!}{k!(n-k)!} \dfrac{k! C_M^k (n-k)! C_{N-M}^{n-k}}{n!C_N^n}
= \dfrac{C_M^k C_{N-M}^{n-k}}{C_N^n}$
\begin{definition}    
    Набір ймовірностей 
    $\left\{ \dfrac{C_M^k C_{N-M}^{n-k}}{C_N^n}, k: \begin{array}{l}
        0 \leqslant k \leqslant M \\
        0 \leqslant n-k \leqslant N-M \\
    \end{array} \right\}$ --- це \textbf{гіпергеометричний розподіл}
    з параметрами $N$, $M$, $n$.
\end{definition}

Можна показати, що при $N \rightarrow \infty$, $M \rightarrow \infty$
так, що $\dfrac{M}{N} \rightarrow$:

$$\dfrac{C_M^k C_{N-M}^{n-k}}{C_N^n} \xrightarrow[\begin{matrix}
    \scriptstyle N \rightarrow \infty\\
    \scriptstyle M \rightarrow \infty\\
    \scriptstyle \frac{M}{N} \rightarrow p\\
\end{matrix}]{} C_n^k p^k(1-p)^{n-k}$$

\section{Статистика Больцмана, Бозе-Ейнштейна, Фермі-Дірака}

\begin{problem}
    Є $n$ частинок, кожна з яких може знаходитися з однаковою ймовірністю
    $\frac{1}{N}$ в кожній з $N$ комірок ($N > n$). Потрібно
    знайти, наприклад, ймовірність того, що певна комірка виявиться порожньою,
    всі частинки потрапляють в різні комірки, тощо.
\end{problem}
\begin{enumerate}
    \item В статистиці Больцмана рівноімовірними є довільні розміщення, що
    відрізняються не лише кількістю, а й набором частинок в комірці.
    
    $\Omega_1 = \{(\varepsilon_1, ..., \varepsilon_n), \varepsilon_i \in \{1, ..., N\}\}$,
    
    $\varepsilon_i$ --- це номер комірки, яку $"$вибрала$"$ собі $i$-та частинка.
    
    $|\Omega_1| = N^n$, $P(\omega) = \frac{1}{N^n} \quad \forall \omega \in \Omega_1$
    
    \item В статистиці Бозе-Ейнштейна вважаються тотожними випадки, коли
    частинки міняються місцями між комірками: важлива лише кількість
    частинок в комірці. Тоді
    
    $\Omega_2 = \{ \omega: \omega = (\varepsilon_1, ..., \varepsilon_n): \varepsilon_i \geqslant 0,
    \sum\limits_{i=1}^N \varepsilon_i = n \}$,
    
    $\varepsilon_i$ --- кількість частинок в $i$-тій урні.
    
    $|\Omega_2| = C_{N+n-1}^n$, $P(\omega) = \frac{1}{C_{N+n-1}^n}$.
    
    \item Згідно статистики Фермі-Дірака в кожній комірці може знаходитись
    не більше однієї частинки.
    
    $\Omega_3 = \{ \omega: \omega = (\varepsilon_1, ..., \varepsilon_n),
    \varepsilon_1 \neq \varepsilon_2 \neq ... \neq \varepsilon_n,
    \varepsilon_i \in \{1, ..., N\}\}$,
    
    $|\Omega_3| = N(N-1)...(N-n+1) = A_N^n$,
    
    $\forall \omega \in \Omega_3: P(\omega) = \frac{1}{A_N^n}$.
\end{enumerate}
    
    
    
    