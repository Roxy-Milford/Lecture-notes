\chapter{Вступ в теорію імовірності}

\section{Математична модель імовірнісного експерименту}

\begin{definition}
    \textbf{Ймовірнісним (статистичним) експериментом} ---
    називають експеримент, для якого:

    1) Множин а можливих результатів наперед відома.

    2) Наперед знати, яким саме результатом закінчиться
    експеримент, ми не можемо.

    3) Експеримент можна повторювати як завгодно багато
    разів при однакових умовах.
\end{definition}

\begin{example}~\par
    1) Підкидання монети.

    2) Підкидання кубика.
\end{example}

\begin{definition}
    \textbf{Простором елементарних подій} експерименту,
    називають множину $\Omega$ всіх можливих результатів
    експерименту.
\end{definition}

\begin{example}~\par
    1) Монета підкидається 1 раз.

    $\Omega = \{\omega_1, \omega_2\} 
    = \{\text{Р}, \text{Г}\}; |\Omega| = 2$

    2) Монета підкидається двіччі.

    $\Omega = \{\text{ГГ}, \text{ГР}, \text{РГ}, \text{РР}\}; |\Omega| = 4$

    3) Монету підкидають до першої появи герба.

    $\Omega = \{\text{Г}, \text{РГ}, \text{РРГ}, ...\}$
    
    $\omega_i = \underbrace{\text{РР...Р}}\limits_{i-1}\text{Г}, i = 1, 2, 3, ...$
    
    $|\Omega| = \infty, \Omega \text{ --- зліченна множина}$

    4) Задача про зустріч.

    Двоє людей домовились про зустріч в секретному місці
    між 12:00 і 13:00. Кожен з них вибирає час приходу на
    місце навмання.

    $\Omega = \{(t_1, t_2): 0 \leqslant t_1 \leqslant 1,
    0 \leqslant t_2 \leqslant 1\}$

    $t_1$ --- час прибуття 1-ої людини

    $t_1$ --- час прибуття 2-ої людини

    $\Omega$ має потужність континума.
\end{example}

\section{Дискретний простір елементарних подій}

Вважаємо, що $\Omega$ --- скінченна або зліченна множина.

 \begin{definition}
    Довільна підмножина $A \subseteq \Omega$ дискретного
    простору елементарних подій --- це \textbf{випадкова
    подія}.
 \end{definition}

 \begin{remark}
    Якщо $\Omega$ --- довільна (не обов'язково дискретна),
    то взагалі кажучи не кожна її підмножина є випадковою
    подією.
 \end{remark}

 Кажуть, що подія $A$ відбулася, якщо відбулася якась з 
 елементарних подій $\omega \in A$.

 \begin{example}~\par
    1) Підкидають 1 раз гральний кубик.\par
    $\Omega = \{1, 2, 3, 4, 5,6\}; |\Omega| = 6$\par
    $A = \{$Випала парна кількість очок$\}$\par
    $A = \{2, 4, 6\}, |A| = 3$

    2) Кидок 2-х гральних кубиків.\par
    $\Omega = \{(i,j), i=\overline{1,6};
        j=\overline{1,6}\}; |\Omega| = 6 \cdot 6 = 36$\par
    $A = \{$сума очок дорівнює 4$\}$\par
    $A = \{(1,3), (2,2), (3,1)\}, |A| = 3$

    3) Підкидання монети до першого герба.\par
    $\Omega = \{$Г, РГ, РРГ, ...$\}$\par
    $A = \{$Було проведено непарну кількість підкидань$\}$\par
    $A = \{\text{Г}, \text{РРГ}, ..., \underbrace{\text{Р}\cdots\text{Р}}\limits_{2k}\text{Г}, ...\}$
    $A$ --- зліченна.

    \section{Операції над подіями}

    1) Обєднання (сума) подій $A$ і $B$ --- це подія
    $A \cup B = \{\omega \in \Omega: \omega in A \vee \omega \in B\}$

    2) Перетин (добуток) подій $A$ і $B$ --- це подія
    $A \cap B = \{\omega \in \Omega: \omega in A \wedge \omega \in B\}$
    --- відбулися обидві події.

    3) $A \setminus B = \{\omega \in \Omega: \omega in A \wedge \omega \notin B\}$
    --- відбулась подія $A$, але не відбулася подія $B$.

    $$A \setminus B = A \cap \underline{B}$$

    4) $A \subset B$ --- з події $A$ випливає подія $B$.
 \end{example}

\begin{definition}~\par
    1) Подія $\Omega$ називається достовірною подією.
    
    2) Подія $\varnothing \subset \Omega$ називається неможливою
    подією.

    3) Подія $\overline{A} = \Omega \setminus A$ називається
    протилежною до події $A$.

    4) Події $A$ і $B$ називаються несумісними, якщо
    $A \cap B = \varnothing$
\end{definition}

\begin{example}
    Підкидають два гральних кубика.

    $\Omega = \{(i,j), i=\overline{1,6}, j=\overline{1,6}\}$

    $A = \{$сума очок дорівнює 4$\}$

    $A = \{(1,3), (2,2), (3,1)\}$

    $B = \{$на першому кубику 6 очок$\}$

    $B = \{(6,j), j = \overline{1,6}\}$

    $A \cap B = \varnothing \Rightarrow A$ і $B$ несумісні.
\end{example}

\begin{remark}
    До подій, як до множини, можна застосувати правила де
    моргана:

    а) $\overline{\bigcup\limits_{i=1}^n A_i}
    = \bigcap\limits_{i=1}^n \overline{A_i}$

    б) $\overline{\bigcap\limits_{i=1}^n A_i}
    = \bigcup\limits_{i=1}^n \overline{A_i}$
\end{remark}

\section{Ймовірність випадкової події}

Вважаємо, що $\Omega$ --- дискретна множина. Кажуть, що на 
$\Omega$ задано розподіл ймовірностей, якщо кожній елементарній
події $\omega \in \Omega$ ставиться у відповідність число
$P(\omega)$ так, що:

1) $\forall \omega \in \Omega: P(\omega) \geqslant 0$

2) $\sum\limits_{\omega \in \Omega} P(\omega) = 1$ --- умова
нормування.

Тоді для довільної випадкової події $A \subseteq \Omega:$

$$P(A) = \sum\limits_{\omega \in A} P(\omega)$$

$P(A)$ називається ймовірністю події $A$.

\subsection*{Властивості ймовірності:}~

1) $P(\Omega) = 1$ --- випливає з умови нормування, (2)

2) $P(\varnothing) = 0$

3) $P(A \cup B)
= \sum\limits_{\omega \in A \cup B} P(\omega)
= \sum\limits_{\omega \in A} P(\omega)
    + \sum\limits_{\omega \in B} P(\omega)
    - \sum\limits_{\omega \in A \cap B} P(\omega)
= P(A) + P(B) - P(A \cap B)$

Для несумісних подій ($A \cap B = \varnothing$)
$P(A \cup B) = P(A) + P(B)$

4) $P(\overline{A})
= \sum\limits_{\omega \in \overline{A}} P(\omega)
= \sum\limits_{\omega \in \Omega \setminus A} P(\omega)
= 1 - P(A)$

\section{Класична модель}

Якщо $|\Omega| = N$ і всі елементарні події вважаються
рівноімовірними, тобто

$P(\omega_1) = ... = P(\omega_N) = \dfrac{1}{N}$,

то $\forall A \subseteq \Omega$:

$P(A)
= \sum\limits_{\omega \in A} \dfrac{1}{N}
= \dfrac{1}{N} |A|
= \dfrac{|A|}{|\Omega|}$

--- відношення кількості сприятливих елементарних подій для
$A$ до загальної кількості елементарних подій.

\begin{example}~\par
    1) Підкидаємо симетричну монету

    $\Omega = \{\text{Р}, \text{Г}\}$

    Нехай $P(\text{Р}) = \dfrac{1}{2}$,
    $P(\text{Г}) = \dfrac{1}{2}$ --- припущення що до
    симетричності монети.

    Якщо $P(\text{Р}) = p$, $P(\text{Г}) = (1-p)$, де
    $p \neq \dfrac{1}{2}$, то монета несиметрична.

    2) Монету підкидають $n$ разів, фіксована кількість.

    $\Omega = \{(\varepsilon_1, \varepsilon_2, ..., \varepsilon_n), \varepsilon_i = 0,1\}$,
    де $\varepsilon_i = 0$, якщо випала решка в $i$-тому
    підкиданні, $\varepsilon_i = 1$, якщо випав герб.

    $|\Omega| = 2^n$

    Покладемо: $P(\omega) = \dfrac{1}{2^n}$ --- припущення
    рівноможливості елементарних подій.

    \begin{remark}
        $\sum\limits_{\omega \in \Omega} \dfrac{1}{2^n}
        = \dfrac{1}{2^n} 2^n = 1$

        $A = \{\text{герб випав } k \text{ разів в } n
        \text{випробуваннях}\}$

        $A = \{(\varepsilon_1, ..., \varepsilon_n) \in \Omega:
        \sum\limits_{i = 1}^n \varepsilon_i = k\}$

        $|A| = C_n^k$ --- кількість способів вибрати $k$-елементну
        підмножину $n$-елементної множини

        Таким чином: $P(A) = \dfrac{|A|}{|\Omega|}
        = \dfrac{C_n^k}{2^n}$

    \end{remark}

    3) Два гравці почерзі підкидають монету. Виграє той, в кого
    раніше випаде герб.

    $A = \{$ виграв 1-ший гравець $\}$

    $B = \overline{A} = \{$ виграв 2-гий гравець $\}$

    $\Omega = \{$Г, РГ, РРГ, ...$\}$

    $A = \{$Г, РРГ, РРРРГ, ...$\}$

    $B = \{$РГ, РРРГ, ...$\}$

    Покажимо: $\omega_i = \underbrace{\text{РРР...Р}}\limits_{i-1}\text{Г}$,
    $i = 1, 2, ...$

    Нехай $P(\omega_i) = \dfrac{1}{2^i}$

    Маємо: $\sum\limits_{i=1}^{\infty} \dfrac{1}{2^i}
    = \dfrac{\dfrac{1}{2}}{1 - \dfrac{1}{2}} = 1$ --- умова
    нормування виконується.

    Згідно значення:

    $P(A) = \sum\limits_{i-\text{непарне}} P(\omega_i)$,
    $P(B) = 1 - P(A) = \sum\limits_{i-\text{парне}} P(\omega_i)$

    $P(A) = \sum\limits_{k=0}^{\infty} \dfrac{1}{2^{2k+1}}
    = \dfrac{1}{2} \sum\limits_{k=0}^{\infty} (\dfrac{1}{4})^k
    = \dfrac{1}{2} \dfrac{1}{1 - \dfrac{1}{4}} = \dfrac{2}{3}$

    Тоді: $P(B) = 1 - P(A) = \dfrac{1}{3}$.

    4) Задача про секретаря

    Написано $n$ листів різним адресатам. Ці листи навмання
    вкладаються в конверти з адресами.

    $A = \{$Принаймі один лист прийде за призначенням$\}$.

    Опишемо простір елементарних подій.

    $\Omega = \left\{ \begin{pmatrix}
        1 & 2 & ... & n \\
        i_1 & i_2 & ... & i_n \\
    \end{pmatrix}, i_j = \overline{1,n}; i_j - \text{різні} \right\}$
    --- множина перестановок множини $(1, 2, ..., n)$.

    Подія $A$ зображається у вигляді
    
    $A = A_1 \cup A_2 ... \cup A_n$,

    де $A_i = \{i$-тий лист прийшов за призначенням$\}$

    $A$-відбулася принаймі одна з подій $A_1, ..., A_n$.

    $P(A) = P(A_1 \cup A_2 ... \cup A_n) = ?$
\end{example}

\begin{lemma}[Формула включень-виключень]
    Нехай $A_1, ..., A_n$ --- випадкові події а $\Omega$. Тоді
    $P(\bigcup\limits_{i=1}^n A_i)
    = \sum\limits_{i=1}^n P(A_i)
        - \sum\limits_{i<j} P(A_i \cap A_j)
        + \sum\limits_{i<j<k} P(A_i \cap A_j \cap A_k) - ...
        + (-1)^{n-1} P(A_1 \cap A_2 \cap ... \cap A_n)$
\end{lemma}





