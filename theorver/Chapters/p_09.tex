\chapter{Неперервні випадкові величини. Основні ймовірнісні розподіли}

\section{Нехай $X$-випадкова величина з неперервною функцією розподілу $F(x) = P(X \leqslant x), x in \mathbb{R}$}

Якщо існує невідємна інтегровна на $\mathbb{R}$ функція
$f(x), x \in \mathbb{R}$ така, що

\begin{equation}
    \forall x \in \mathbb{R} \quad F(x) = \int\limits_{-\infty}^{x} f(y) dy,
\end{equation}

то функція $f$ --- це \textbf{щільність розподілу випадкової величини $X$},
а сама випадкова величина $X$ --- це \textbf{неперервна величина}.

З означення випливає, що для неперервної випадкової величини $X$ справедливі такі рівності:

1) \begin{equation}
    \int\limits_{-\infty}^{\infty} f(x) dx = 1
\end{equation}

Умова нормування () випливає з того, що $F(\infty) = 1$.

2) \begin{equation}
    P(a < X \leqslant b) = \int\limits_{a}^{b} f(x) dx
\end{equation}

Рівність () випливає з того, що 

$P(a < X \leqslant b) = F(b) - F(a)
= \int\limits_{-\infty}^{b} f(x) dx - \int\limits_{-\infty}^{a} f(x) dx
= \int\limits_{a}^{b} f(x) dx.$

Оскільки функція $F$ є неперервною, то $P(X = x) = 0$ і

\begin{equation}
    P(a < X < b) = P(a \leqslant X \leqslant b)
    = P(a < X \leqslant b)
    = \int\limits_{a}^{b} f(x) dx
\end{equation}

Загалом, для довільної множини $B \in \mathbb{B}$:

\begin{equation}
    P(X \in B) = \int\limits_{B} f(x) dx
\end{equation}

3) З властивості інтеграла зі змінною верхньою межею випливає, що

\begin{equation}
    f(x) = F'(x)
\end{equation}

в точках неперервності функції $f$.

\begin{example}
    Нехай випадкова величина має щільність розподілу
    
    $$f(x) = C (4x -2x^2)\mathbb{I}(0 < x < 2).$$
    
    Знайдемо константу $C$ та ймовірність $P(X > 1)$.
    
    Константу $C$ знаходимо з умови нормування:
    
    $$1 = \int\limits_{-\infty}^{\infty} f(x) dx
    = C \int\limits_{0}^{2} (4x -2x^2) dx
    = C (4 \frac{2^2}{2} - 2 \frac{2^3}{3})
    = \frac{8}{3} C \Rightarrow c = \frac{3}{8}$$
    
    Тоді

    $$P(X > 1) = \int\limits_{1}^{\infty} f(x) dx
    = \frac{3}{8} \int\limits_{0}^{2} (4x -2x^2) dx
    = \frac{1}{2}$$
\end{example}

\section{Рівноімовірний розполіл на відрізку}

\begin{definition}
    
\end{definition}
Випадкова величина $X$ має \textbf{рівноімовірний розподіл на відрізку $[a, b]$},
якщо її щільність розподілу має вигляд

\begin{equation}
    \label{rivnoymovirnyi_rozpodil}
    f(x) = \frac{1}{b-a} \mathbb{I}(x \in [a, b])
\end{equation}

з (\ref{rivnoymovirnyi_rozpodil}) випливає, що функція розподілу $X$ має вигляд

\begin{equation}
    F(x) = \int\limits_{-\infty}^{x} f(y) dy
    = \left\{\begin{array}{ll}
        0, & x < a \\
        \frac{x-a}{b-a}, & x \in [a, b] \\
        1, & x > b \\
    \end{array}\right.
\end{equation}

\beautifulImage

\begin{example}
    Нехай $X \sim U[0, 1]$, тобто $X$ має рівномірний розподіл на $[0, 1]$.
    Тоді $f(x) = \mathbb{I}(x \in [0, 1])$ і отже
    $P(\frac{1}{3} < X < \frac{1}{2}) = \int\limits_{\frac{1}{3}}^{\frac{1}{2}} dx = \frac{1}{6} = \frac{l_A}{l_{\Omega}}$, 
    
    де $\Omega = [0, 1], A = (\frac{1}{3}, \frac{1}{2})$.
    
    \beautifulImage
\end{example}

Нехай $X \sim U[0, 1]$. За допомогою $X$ можна побудувати випадкову величину з
заданою неперервною функцією розподілу $F$. Справді, покладемо $Y = F^{-1}(X)$.

Тоді 

$$F_Y(x) = P(Y \leqslant x)
= P(F^{-1}(X) \leqslant x)
= P(X \leqslant F(x))
= F(x),$$

оскільки $X \sim U[0, 1]$ і $0 \leqslant F(x) \leqslant 1$.

Навпаки, маючи випадкову величину $Y$ з неперервною функцією розподілу
$F$, можна побудувати випадкову величину з рівномірним на $[0, 1]$
розподілом.

Справді, поклавши $X = F(Y)$, отримаємо для $x \in [0, 1]$:

$$F_X(x) = P(X \leqslant x)
= P(F(Y) \leqslant x)
= P(Y \leqslant F^{-1}(x))
= F(F^{-1}(x))
= x$$

При $x < 0: \quad P(F(Y) \leqslant x) = 0$

При $x > 1: \quad P(F(Y) \leqslant x) = 1$

Отже $X \sim U[0, 1]$.

\section{Нормальний розподіл}

Випадкова величина $X$ має нормальний розподіл з параметрами $a \in \mathbb{R}$
і $\sigma^2$ $(\sigma > 0)$, що записується як $X \sim N(a, \sigma^2)$,
якщо її щільність розподілу має вигляд:

\begin{equation}
    f_{a, \sigma^2}(x) = \frac{1}{\sqrt{1\pi} \sigma} e^{-\frac{(x-a)^2}{2\sigma^2}},
    x \in \mathbb{R}
\end{equation}

Якщо $a = 0$, $\sigma^2 = 1$, то кажуть, що $X$ має \textbf{стандартний нормальний розполіл}.

Функція розподілу випадкової величини $X$ позначається $\Phi$ і

\begin{equation}
    \Phi_{a, \sigma^2}(x)
    = \frac{1}{\sqrt{1\pi} \sigma} \int\limits_{-\infty}^{x} e^{-\frac{(y-a)^2}{2\sigma^2}} dy,
    x \in \mathbb{R}
\end{equation}

Оскільки $\Phi_{a, \sigma^2}(x) = \Phi_{0, 1}(\frac{x-a}{\sigma})$,
то $a$-параметр зсуву, $\sigma$-параметр маштабу.

Графік функції $f(x)$ є симетричним відносно $x = a$.

\beautifulImage

Вперше нормальний розподіл було використано в 1733 році для
апроксимації біноміального розподілу, коли $n$ є великим.

\begin{theorem}[Локальна теорема Муавра]
    \begin{equation}
        \sqrt{n p (1-p)} p_n(k) \approx f_{0, 1}(x),
    \end{equation}

    де $p_n(k) = C_n^kp^k(1-p)^{n-k}$,
    $x = \dfrac{k-np}{\sqrt{n p (1-p)}}$,
    $k = \overline{0, n}$
    При великому $n$.
\end{theorem}

\begin{example}
    Нехай ймовірність влучених в мішень при одному пострілі дорівнює 0.8.
    Нехай $X$ --- кількість влучень при $n = 400$ пострілах. Обчислити
    $P(X = 300)$.

    $P(X = 300)
    = P_{400}(X = 300)
    = C_{400}^{300} (0.8)^{300} (0.2)^{100}$
    
    Скористаємося нормальним наближенням:
    
    $$\sqrt{n p (1-p)} = \sqrt{400 \cdot 0.8 \cdot 0.2} = 8,$$
    
    $$x = \dfrac{k-np}{\sqrt{n p (1-p)}} = \dfrac{300-320}{8} = - 2.5,$$
    
    $$P(X = 300) \approx \frac{1}{8} f_{0,1}(-2.5) = 0.0022.$$
\end{example}
    
\subsection*{Інтегральна формула Муавра-Лапласа:}

Якщо $X \sim Bin(n, p)$, причому $n$ є великим і ймовірномті $p$
і $(1-p)$ є не надто малим, то

\begin{equation}
    P(K_1 \leqslant X \leqslant K_2)
    \approx \Phi_{0,1}(x_2) - \Phi_{0,1}(x_1)
    = \inf\limits_{x_1}^{x_2} f_{0,1}(x)dx
\end{equation}

де $\Phi_{0,1}$-функція розподілу стандартного нормального розподілу,
$x_1 = \frac{k_1 - np}{\sqrt{np(1-p)}}$, $x_1 = \frac{k_2 - np}{\sqrt{np(1-p)}}$

\begin{example}
    Нехай $X$-кількість шісток, що випали при $n = 600$ підкиданнях грального кубика.
    Обчислимо наближено ймовірність $P(90 \leqslant X \leqslant 120)$:

    $P(90 \leqslant X \leqslant 120)
    \approx \Phi_{0,1}(\frac{120 - 600 \frac{1}{6}}{\sqrt{600 \frac{1}{6} \frac{5}{6}}})
        - \Phi_{0,1}(\frac{90 - 600 \frac{1}{6}}{\sqrt{600 \frac{1}{6} \frac{5}{6}}})
    = \Phi_{0,1}(2.19) - \Phi_{0,1}(-1.1)
    =0.85007$
\end{example}

\section{Показниковий (Експоненціальний) розподіл, $Exp(\lambda)$}

Неперервна випадкова величина $X$ має \textbf{показниковий розподіл з параметром $\lambda > 0$},
якщо її щільність розподілу має вигляд:

\begin{equation}
    f(x) = \left\{\begin{array}{ll}
        \lambda e^{-\lambda x}, & x \geqslant 0 \\
        0, & x < 0 \\
    \end{array}\right.
    = \lambda e^{-\lambda x} \mathbb{I}(x \geqslant 0)
\end{equation}

Функція Розподілу $X$ має вигляд:

\begin{equation}
    F(x) = P(X \leqslant x)
    = \int\limits_{-\infty}^{x} f(y) dy
    = \left\{\begin{array}{ll}
        0, & x < 0 \\
        \int\limits_{0}^{x}\lambda e^{-\lambda x} dy = 1 - e^{1-\lambda x}, & x \geqslant 0 \\
    \end{array}\right.
\end{equation}

Показниковий розподіл широко застосовується при моделювання часу безвідмовної роботи пристроїв,
часу між надходженнями вимог в симтемах масового обслуговування і тривалості обслуговування цих
вимог.

\begin{claim}[Відсутність післядії]
    Якщо $X \sim Exp(\lambda)$, то для довільних $s, t \geqslant 0$

    \begin{equation}
        P(X > s + t \mid X > s) = P(X > t)
    \end{equation}
\end{claim}
\begin{proof}
    Зауважимо, що
    $$P(X > t)
    = \int\limits_{t}^{\infty} f(x) dx
    = \int\limits_{t}^{\infty} \lambda e^{-\lambda x} dx
    = e^{-t}.$$

    За означенням умовної ймовірності

    $$P(X > t + s \mid X > s)
    = \frac{P(X > t + s, X > s)}{P(X > s)}
    = \frac{P(X > t + s)}{P(X > s)}.$$
    
    Тоді
    
    $$P(X > t + s \mid X > s)
    = \frac{e^{-(t+s)}}{e^{-s}}
    = e^{-t},$$
    
    що й потрібно було довести.
\end{proof}

\beautifulImage

\section{Розподіл Коші $C(\alpha, \beta)$}

Непрервна випадкова величина $X$ має \textbf{розподіл Коші $C(\alpha, \beta)$},
якщо її щільність розподілу має вигляд

\begin{equation}
    f(x) = \frac{1}{\pi} \frac{\beta}{(x-\alpha)^2 + \beta^2}
\end{equation}


При $\alpha = 0$, $\beta = 1$ говорять про \textbf{стандартний розподіл Коші}.
Цей розподіл виникає, наприклад, в наступній задачі:

\begin{example}
    Нехай в точці $(0,1)$ в $\mathbb{R}^2$ Поміщено джерело випромінювання
    частинок. Детектор, який співпадає з віссю $Ox$,
    фіксує сліди випромінювання.
\end{example}

Напрямок випромінювання є випадковим і має рівномірний розподіл на $[-\pi, \pi]$.

Нехай $X$-кордината, в якій детектор зафіксував частинку.

\beautifulImage

$\varphi$-кут між напрямком руху частинки і віссю $Oy$.

Частинка досягає детектора, якщо $\varphi \sim U{-\frac{\pi}{2}, \frac{\pi}{2}}$.

$X = \tg \varphi$

Знайдемо функцію розподілу $X = \tg \varphi$:

$F(x)
= P(X \leqslant x)
= P(\tg \varphi \leqslant x)
= P(\varphi \leqslant \arctg x)
= \frac{\arctg x - (-\frac{\pi}{2})}{\frac{\pi}{2} - (-\frac{\pi}{2})}
= \frac{1}{2} - \frac{1}{\pi} \arctg x$

Оскільки $(\arctg x)' = \frac{1}{1+x^2}$,

то

$$\arctg x
= \int\limits_0^x \frac{dy}{1+y^2}
= \int\limits_{-\infty}^x \frac{dy}{1+y^2} - \frac{\pi}{2}$$

і Тоді

$$P(X \leqslant x)
= \frac{1}{\pi} \int\limits_{-\infty}^x \frac{dy}{1+y^2},$$

а отже

$$f(x) = \frac{1}{\pi} \frac{1}{1 + y^2}.$$

\section{Розподіл багатовимірної випадкової величини}

Нехай $(\Omega, \mathcal{F}, P)$ --- ймовірнісний простір, $X_1$, $X_2$, ...,
$X_n$ --- випадкові величини на $(\Omega, \mathcal{F}, P)$.

Впорядкований набір $X = (X_1, X_2, ..., X_n)$ --- це
\textbf{$n$-вимірний випадковий вектор}, або \textbf{$n$-вимірна випадкова
величина}. $X_i$ --- $i$-та кордината.

\begin{example}    
    Нехай $(X_1, X_2, ..., X_n)$ --- це координати відхилення зенітного
    снаряда від точки прицілювання в деякій просторовій системі координат.
\end{example}

Функція вигляду:

\begin{equation}
    F(x_1, x_2, .., x_n)
    = P(X_1 \leqslant, X_2 \leqslant, ..., X_n \leqslant),
    (x_1, x_2, .., x_n) \in \mathbb{R}^n
\end{equation}

--- це \textbf{функція розподілу випадкового вектора $X = (X_1, X_2, ..., X_n)$}.

При $n = 2$ маємо двовимірну випадкову величину $X = (X_1, X_2)$, для якої 

$$F(a, b)
= P(X_1 \leqslant a, X_2 \leqslant b)$$

задає ймовірність потрапляння $(X_1, X_2)$ в заштриховану область

\beautifulImage

Властивості функції розподілу (при $n = 2$).

\begin{theorem}Двовимірна функція розподілу $F(x, y)$ задовольняє таким умовам:
    \begin{enumerate}
        \item $0 \leqslant F(x, y) \leqslant 1$,
        \item $F(x, y)$ неспадна за кожним аргументом
        \item $F(-\infty, y) = (x, -\infty) = 0$
        \item $F(\infty, \infty) = 1$
        \item $F(x, y)$ неперервна справа в довільній точці $(x, y) \in \mathbb{R}^2$
        за кожниим з аргументів
        \item $P(a_1 \leqslant X_1 \leqslant b_1, a_2 \leqslant X_2 \leqslant b_2)
        = F(b_1, b_2) - F(a_1, b_2) - F(b_1, a_2) + F(a_1, a_2)$
        \item $F_{(X_1, X_2)}(x, \infty) = F_{X_1}(x)$
        \item $F_{(X_1, X_2)}(\infty, y) = F_{X_2}(y)$
    \end{enumerate}
\end{theorem}

Якщо всі $X_1$, $X_2$, ..., $X_n$ є дискретними випадковими величинами,
то вектор $X = (X_1, X_2, ..., X_n)$ --- це \textbf{дискретний випадковий вектор}.
Закон розподілу двовиміргого дискретного вектора $(X, Y)$ можна задати таблицю
такого вигляду

\begin{center}
    \begin{tabular}{c|ccccc}
        $x \backslash y$ & $y_1$ & $y_2$ & $y_3$ & $y_4$ & ... \\
        \hline $x_1$ & $p_{11}$ & $p_{12}$ & $p_{13}$ & $p_{14}$ & ... \\
        $x_2$ & $p_{11}$ & $p_{12}$ & $p_{13}$ & $p_{14}$ & ... \\
        $x_3$ & $p_{11}$ & $p_{12}$ & $p_{13}$ & $p_{14}$ & ... \\
        $\vdots$ & $\vdots$ & $\vdots$ & $\vdots$ & $\vdots$ & $\ldots$ \\
    \end{tabular}
\end{center}

де $\{x_1, x_2, ...\}$, $\{y_1, y_2, ...\}$ --- можливі значення випадкової
величини $X$ та $Y$, $p_{ij} = P(X = x_i, Y = y_j)$ --- ймовірність одночасної
появи подій $\{X = x_i\}$ та $\{Y = y_i\}$.

При цьому за двовимірним розподілом можимо знайти одновимірні розподіли
випадкових величин $X, Y$:

\begin{equation}
    \begin{array}{c}
        P(X = x_i) = \sum\limits_j P(X = x_i, Y = y_j) = \sum\limits_j p_{ij} \\
        P(Y = y_i) = \sum\limits_i P(X = x_i, Y = y_j) = \sum\limits_i p_{ij} \\
    \end{array}
\end{equation}

\begin{example}(Поліноміальний Розподіл)
    Нехай проводяться $n$ незалежних випробувань, в кожному з яких
    спостерігається одна з $m$ несумісних подій $A_1$, ..., $A_m$,
    причому $P(A_i) = p_i$, $\sum\limits_{i = 1}^m p_i = 1$.
    
\end{example}


Нехай $X_i$ --- це кількість випробувань, в яких спостерігалася подія $A_i$.
Тоді $X = (X_1, ..., X_m)$ --- дискретний випадковий вектор з розподілом:

\begin{equation}
    P(X_1 = K_1, ..., X_m = K_m)
    = \frac{n!}{K_1!K_2!...K_m!} p_1^{K_1} ... p_m^{K_m},
\end{equation}

де $K_i \geqslant 0$ і $\sum\limits_{i = 1}^m K_i = n$.

Якщо $m = 2$, то отримаємо біноміальний розподіл.

Розподіл вектора $X = (X_1, ..., X_n)$ --- це \textbf{неперервний розподіл},
якщо існує невад'ємна інтегровна на $\mathbb{R}^n$ функція $f(x_1, ..., x_n)$ така, що

\begin{equation}
    \forall B \in \mathcal(B)(\mathbb{R}^n): P(X \in B ) = \int\limits_B f(x_1, ..., x_n)dx_1...dx_2
\end{equation}

Умова () еквівалентна тому, що 

\begin{equation}
    F(x_1, ..., x_n)
    = \int\limits_{-\infty}^{x_1} ... \int\limits_{-\infty}^{x_n}
        f(y_1, ..., y_n)dy_1...dy_n
\end{equation}

для довільного $x = (x_1, ..., x_n) \in \mathbb{R}^n$.

Функцію $f(x_1, ..., x_n)$ з (), () --- це \textbf{щільність розпоілу випадкового вектора $X$}.
З властивостей функції розподілу $F$ і властивостей інтеграла зі змінною верхньою мужею випливає, що 

\begin{equation}
    \int\limits_{-\infty}^{\infty} ... \int\limits_{-\infty}^{\infty}
        f(x_1, ..., x_n)dx_1...dx_n
    = 1
\end{equation}

і

\begin{equation}
    f(x_1, ..., c_n)
    = \frac{\delta^n F(x_1, ..., x_n)}{\delta x_1, .., \delta x_n}
\end{equation}

Зауважимо, що за щільністю $f(x_1, ..., x_n)$ випадкового вектора $X = (X_1, .., X_n)$м
ожна знайти щільність кожної координати. Справді:

$$F_{X_i}(x_i)=
F_{(X_1, ..., X_i, ..., X_n)(\infty, ..., x_i, ..., \infty)}
= \int\limits_{-\infty}^{x_i} \int\limits_{-\infty}^{\infty} ... \int\limits_{-\infty}^{\infty}
    f(y_1, ..., y_n) dy_1 ... dy_n$$

отже

\begin{equation}
    f_{X_i}(x_i) = \underbrace{\int\limits_{-\infty}^{\infty} ... \int\limits_{-\infty}^{\infty}}_{n-1}
    f(x_1, ..., x_n) dx_1 ... dx_{i-1} dx_{i+1} ... dx_n
\end{equation}

\begin{example}(рівномірний розподіл в крузі)

\end{example}

Нехай $\Omega = \{(x, y) \in \mathbb{R}^2: x^2 + y^2 \leqslant r^2\}$ і нехай 
$(X, Y)$ --- координати навмання вибраної точки в $\Omega$. Тоді для довільної
підмножини $B \subseteq \Omega$:

$$P((X, Y) \in B)
= \frac{S_B}{\pi r^2}
= \iint\limits_{B} \frac{dx dy}{\pi r^2}$$

Звідси 

$$f_{(X, Y)} (x, y)
= \frac{1}{\pi r^2} \mathbb{I}(x^2 + y^2 \leqslant r^2).$$

Знайдемо одновимірні щільності $f_X(x)$, $f_Y(y)$.

Маємо:

$$f_X(x)
= \int_{-\infty}^{\infty} f_{(X, Y)}(x, y) dy
= \frac{1}{\pi r^2} \int_{-\infty}^{\infty} \mathbb{I}(x^2 + y^2 \leqslant r^2) dy
= \frac{1}{\pi r^2} \int_{-\sqrt{r^2 - x^2}}^{\sqrt{r^2 - x^2}} dy
= \frac{2}{\pi r^2}\sqrt{r^2 - x^2}, \text{ при } |x| \leqslant r$$

$$f_X(x) = 0, \text{ при } |x| > r.$$

Аналогічно,

$$f_Y(y)
= \frac{2}{\pi r^2}\sqrt{r^2 - y^2} \mathbb{I}(|y| \leqslant r).$$

Приклад: Нехай двовимірна випадкова величина  має щільність розподілу

Обчислимо ймовірності  .

Оскільки 

то 


Далі,

, де

Тому 

7. Незалеєність випадкових величин

Випадкові величини , називаються незалежними, якщо 

для довільних , .

теорема (Критерій незалежності неперервних випадкових величин)

Неперервні випадкові величини , зі щільностями  та  відповідно є незалежними тоді 
і лише тоді, коли вектор  має щільність розподілу  для якої

Якщо  і  є незалежними, то  

Звідси 

а з іншого боку


Отже 

Якщо  , то 

що означає незалежність  та  .

Приклад 
Нехай  , - незалежні випадкові величини, кожна з яких має показниковий розподіл з параметрами  ,  
відповідно.
Покажемо, що випадкова величина   маж показниковий розподіл з параметром  .

Справді, для довільного  маємо:


незалежність

Отримати функцію розподілу випадкової величини з параметром .

Поняття незалежності поширюється на довільну кількість випадкових  .

Випадкові величини  називаються незалежними, якщо для довільних  :


Приклад Якщо  незалежні випадкові величини і  , тобто

(Довести самостійно)

Зауваження Існують також так звані стнгулярні випадкові величини, які не є ні дискретними, ні неперервними.

Приклад: Нехай  - рівномірно розподілена на  випадкова величина. Покладемо 

Ця випадкова величина не є ні дискретною, ні неперервною.

Функція розподілу  є розривною в точці :

, отже не є неперервною.

На інтервалі  є неперервною функцією









