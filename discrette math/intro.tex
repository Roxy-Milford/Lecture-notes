\chapter{Вступ}

Аналітична геометрія – це розділ геометрії, основними об’єктами вивчення
якого є точки, прямі, криві, площини та поверхні.

Основним методом вивчення є метод координат, який виклав Р.Декарт у
1637 р. Суттєвий внесок у розвиток методу координат зробив і його сучасник
П.Ферма. Суть методу полягає в тому, що кожній точці у просторі ставиться у
відповідність трійка чисел – її координати. Далі геометричні об’єкти
досліджуються за допомогою алгебраїчних рівнянь відносно цих координат.

Основна мета методу – це побудова такої системи координат, в якій лінії,
поверхні тощо мали б найпростіший вигляд.

Лінійна алгебра – це розділ алгебри, в якому вивчаються векторні (лінійні)
простори, лінійні оператори, функціонали (відображення) на цих просторах.

Історично першим розділом лінійної алгебри була теорія лінійних рівнянь.
Були введені поняття матриці, визначника матриці та рангу матриці.

Г.Крамер запропонував метод розв’язання систем рівнянь у 1750 р., який було
названо його іменем.

Найбільш універсальним є метод Гауса (1849 р.), який може бути
використаним як у випадку, коли система має єдиний розв’язок, так і у випадках,
коли розв’язків немає взагалі чи їх безліч. Поняття рангу матриці було введено
Ф.Фробеніусом у 1877 р.

До кінця XIX ст. було завершено побудову загальної теорії систем лінійних
рівнянь. У XX ст. центральне місце в лінійній алгебрі займало вивчення так званих
лінійних просторів і лінійних перетворень в них.

Розвиток методів аналітичної геометрії та лінійної алгебри пов’язаний з
іменами видатних математиків:

Рене Декарт (1596 – 1650) – французький філософ, математик, фізик, фізіолог;

П’єр Ферма (1601 – 1665) – французький математик, юрист;

Готфрід Вільгельм Лейбніц (1646 – 1716) – німецький філософ, математик, фізик,
мовознавець;

Ісаак Ньютон (1643 – 1727) – англійський математик, механік, астроном, фізик;

Леонард Ейлер (1707 – 1783) – математик, механік, фізик, астроном (російський
вчений швейцарського походження);

Габріель Крамер (1704 – 1752) – швейцарський математик;

Карл Фрідріх Гаус (1777 – 1855) – німецький вчений;

Фердинанд Георг Фробеніус (1849 – 1917) – німецький математик. 
