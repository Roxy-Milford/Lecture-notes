\chapter{Введення в математичний аналіз}

\section{Най простіші логічні символи, квантори}

Для запису тверджень використовують вирази, зв'язані з логічними зв'язками:

--- заперечення $\neg$, Наприклад, $\neg A$

\begin{tabular}{c|c}
    $A$ & $\neg A$ \\
    \hline
    0 & 1 \\
    1 & 0 \\
\end{tabular}

--- конюкція (логічне $"$і$"$) $\wedge, \&$, Наприклад, $A \wedge B$

\begin{tabular}{c|c|c}
    A & B & $A \wedge B$ \\
    \hline
    0 & 0 & 0 \\
    0 & 1 & 0 \\
    1 & 0 & 0 \\
    1 & 1 & 1 \\
\end{tabular}

--- дизюнкція (логічне $"$або$"$) $\vee$, Наприклад, $A \vee B$

\begin{tabular}{c|c|c}
    A & B & $A \vee B$ \\
    \hline
    0 & 0 & 0 \\
    0 & 1 & 1 \\
    1 & 0 & 1 \\
    1 & 1 & 1 \\
\end{tabular}

--- імплікація (логічне слідування) $\Rightarrow$, Наприклад, $A \Rightarrow B$

\begin{tabular}{c|c|c}
    A & B & $A \Rightarrow B$ \\
    \hline
    0 & 0 & 1 \\
    0 & 1 & 1 \\
    1 & 0 & 0 \\
    1 & 1 & 1 \\
\end{tabular}

--- рівносильність (еквівалентність) $\Leftrightarrow$, Наприклад, $A \Leftrightarrow B$

\begin{tabular}{c|c|c}
    A & B & $A \Rightarrow B$ \\
    \hline
    0 & 0 & 1 \\
    0 & 1 & 0 \\
    1 & 0 & 0 \\
    1 & 1 & 1 \\
\end{tabular}

--- ставиться у відповідність $\mapsto$, Наприклад, $A \mapsto B$

--- $"$такий, що$"$ :, Наприклад, $\{f(x): x \in \mathbb{R}\}$

\subsection*{Логічні квантори}

$\forall$ --- квантор загальності (всі, кожний, для всіх, ...)

\begin{example}
    $(\forall x \in \mathbb{R}): x^2 + 1 > 0$
\end{example}

$\exists$ --- квантор існування ($"$існує такий елемент, що$"$, ...)

\begin{example}
    $(\exists x \in \mathbb{R}): x^2 - 1 = 0$
\end{example}

$\exists!$ --- ($"$існує, і при чому єдиний$"$, ...)

\begin{example}
    $(\exists! x \in \mathbb{R}): (x - 1)^2 = 0$
\end{example}

\section{Поняття множини, операції над множинами}

\begin{definition}
    \textbf{Множина} --- це об'єднання (клас, сімейство) деяких об'єктів,
    об'єднаних по певному признаку.
\end{definition}

Цей признак повинен однозначно визначати об'єкти даної множини (наприклад:
множина усіх студентів інститута, множина коренів рівняння $x^2 + 2x + 2 = 0$,
множина усіх натуральних чисел і тому подібне).

\begin{definition}
    об'єкт, що належить множині --- це \textbf{елемент множини}.
\end{definition}

Множина, зазвичай позначається великою буквою латинського алфавіта: $A$, $B$, ...
$X$, $Y$, ..., a їх елементи маленькими: $a$, $b$, ..., $x$, $y$, ...

Якщо елемент $x$ належить множині $X$, то пишуть $x \in X$. Запис $x \notin X$
означає, що елемент $x$ не належить множині $X$.

\begin{definition}
    Множина, що не містить жодного елемента --- це \textbf{порожня множина} ($\varnothing$)
\end{definition}

Елементи множини записують в фігурних дужках, всередині яких вони перечислені (якщо це можливо)
або вказано загальна властивість, що об'єднує всі елемети даної множини.

\begin{example}
    $A = \{1; 3; 15\}$, $B = \{x \in \mathbb{R}: 0 \leqslant x \leqslant 2 \}$
\end{example}

\begin{definition}
    Множина $A$ --- це \textbf{підмножина} множини $B$ ($A \subset B$), якщо
    $(\forall x \in A): x \in B$ (кожний елемент множини $A$ є елементоммножини $B$)
\end{definition}

$\forall A: A \subset A, \varnothing \subset A$

\begin{definition}
    Множини $A$ і $B$ --- \textbf{рівні множини} або ці \textbf{множини співпадають},
    ($A = B$) якщо $A \subset B$ і $B \subset A$. Інакше кажучи, якщо вони складаються
    з ожних і тих самих елементів.
\end{definition}

\begin{definition}
    \textbf{Перетин множин} $A$ і $B$ ($A \cap B$) --- це множина, що складається з
    елементів, що належать одночасно обом множинам: $A \cap B = \{x: x \in A \wedge x \in B\}$.
\end{definition}

\begin{definition}
    \textbf{Різниця множин} $A$ і $B$ ($A \setminus B$) --- це множина, що складається з
    елементів множини $A$, що не належать множині $B$:
    $A \setminus B = \{x: x \in A \wedge x \notin B\}$.
\end{definition}

\begin{definition}
    \textbf{Універсальна множина} $U$ --- це множина, що складається зусіх елементів
    у даному контексті, (наприклад, множина усіх чисел, множина усіх студентів).
\end{definition}

\begin{definition}
    \textbf{Доповнення множини} $A$ ($\overline{A}$) --- це множина, що складається зусіх елементів
    універсальної множини, крім тих, що належать множині $A$. $\overline{A} = \{x: x \notin A\}$
\end{definition}

\begin{definition}
    \textbf{Декартовий добуток множин} $X$ та $Y$ ($\overline{A}$) --- це множина всіх упорядкованих
    пар $(x;y)$, де $x \in X$ та $y \in Y$
\end{definition}

Наприклад, $X = \{\triangle\}$, $Y = \{1;2\}$, $x \times Y = \{(\triangle;1);(\triangle;2)\}$.

Аналогічно визначемо декартів добуток $n$ множин,
$X^n = \underbrace{X \times X \times ... \times X}\limits_{n\text{ раз}}$.

\begin{remark}
    $X \times Y = \varnothing \Leftrightarrow (X = \varnothing) \vee (Y = \varnothing)$.
\end{remark}

