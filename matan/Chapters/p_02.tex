\section{Поняття бінарного відношення. Відношення порядку. Функціональне відношення}

\begin{definition}
    \textbf{Бінарне відношення} --- це деяка підмножина декартового добутку
    двох множин, тобто довільна множина упорядкованих пар:

    $R \subset X \times Y$, $R \subset \{(x,y): x \in X \wedge y \in Y\}$.
\end{definition}

Запис: $(x,y) \in R$, або $xRy$, означає, що $x$ зв'язаний з $y$
відношенням $R$.

Множини перших та других елементів упорядкованих пар, що утворюють деяке відношення,
це --- \textbf{область визначення} і \textbf{область значень} цього відношення
відповідно.

Якщо $R \subset X^2$ (тобто $R \subset X \times X$), то говорять, що відношення
$R$ \textit{задано на множині $X$}.

\begin{definition}
    Відношення $R \subset X^2$ --- це \textbf{відношення часткового порядку}, якщо
    виконуються наступні властивості:
    \begin{enumerate}
        \item $\forall x \in X \: xRx$ (рефлексивність)
        \item $(xRy \wedge yRx) \Rightarrow x = y$ (антисиметричність)
        \item $(xRy \wedge yRz) \Rightarrow xRz$ (транзитивність)
    \end{enumerate}
\end{definition}

Множина $X$, на якій введено відношення часткового порядку $R \subset x^2$
є частково упорядкованою.

Для такого відношення, запис $xRy$ означає, що $y$ слідує за $x$, або $x$ передує $y$
(де $x$ та $y$ --- елементи множини $X$).

\begin{example}
    Відношення частково порядку на $N$:

    \begin{enumerate}
        \item $R$: "$a$ є дільником $b$".
        \item $R$: "$a$ менше рівно $b$".
    \end{enumerate}
\end{example}

Нехай $R \subset X^2$ --- відношення часткового порядку на $X$. Якщо будь які два елементи
$x$ та $y$ множини $X$ при цьому порівнюємі (тобто $xRy$ або $yRx$), то таке відношення ---
це \textbf{відношення порядку}, а множина $X$ --- \textbf{упорядкована множина}.

\begin{example}~

    \begin{enumerate}
        \item $R$: "$a$ менше рівно $b$".
        \item $R$: "$a$ є дільником $b$" --- не відношення порядку, бо наприклад числа 2 та 3
            не є порівнюваними.
    \end{enumerate}
\end{example}

Відношення $R \subset X \times Y$ --- це \textbf{функціональне відношення}, якщо 
$(xRy_1 \wedge xRy_2) \Rightarrow y_1 = y_2$, тобто якщо воно не містить різних пар з
однаковими першими елементами.

\begin{center}
    \fbox{Красиві} \fbox{картинкИ}
\end{center}

\section{Відображення(функція). Образ та прообраз множини. Класифікація відображень}

Нехай $R \subset X \times Y$ --- функціональне відношення. Якщо множина $X$
співпадає з областю візначення цього відношення (тобто $X$ утворено першими
елементами тільки тих упорядкованих пар, які утворюють відношення $R$), то це
функціональне відношення --- це \textit{відображення функції із $X$ \underline{\underline{в}} $y$}

Таким чином, якщо відношення $R \subset X \times Y$ це функція, то для кожного
елемента $x$ із $X$ існує, щей при тому єдиний, елемент $y$ із $Y$, такий, що
$xRy$.

\begin{equation*}
    R \subset X \times Y \text{ - функція } \Rightarrow \forall x \in X \quad \exists! y \in Y: xRy
\end{equation*}

Якщо не тільки $X$ співпадає з областю визначення, а й $Y$ співпадає з множиною
значень, то це відображення --- це \textit{відображення із $X$
\underline{\underline{на}} $y$}

\begin{center}
    \fbox{Красиві} \fbox{картинкИ}
\end{center}

\begin{remark}
    Довільне відображення із $X$ на $Y$ одночасно є відображенням із $X$ в $Y$,
    але не навпаки.
\end{remark}

\begin{definition}
    Відображення множини в себе, тобто із $X$ в $X$ --- це \textbf{оператор}.
\end{definition}

Функції будемо позначати наступним чином:

$$F: X \rightarrow Y, X \xrightarrow{f} Y, X \ni x \xmapsto{f} y \in Y,$$

а якщо із контексту зрозуміло про які $X$ та $Y$ ідеться, то позначають ---
$y = f(x)$ або прость символом $f$.

\begin{example}~

    \begin{enumerate}
        \item Тотожне відображення: $I: X \rightarrow X; \forall x \in X I(x) = x$
        \item Постійне відображення (константа):
            $C: X \rightarrow Y \quad \exists! y \in Y \quad \forall x \in X \quad C(x) = y$
    \end{enumerate}
\end{example}

