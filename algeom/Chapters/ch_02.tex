\chapter{Визначники і лінійна залежність}


\section{Визначники другого і третього порядків}

\begin{definition}
	\textbf{Матриця $A$, розміром $m \times n$} --- це прямокутна таблицю чисел з
	$m$ рядків та $n$ стовпчиків:
	
	$$A = \begin{pmatrix}
		a_{11}	& a_{12}	& ...	& a_{1n}  \\
		a_{21}	& a_{22}	& ...	& a_{2n}  \\
		...		& ...		& ...	& ...     \\
		a_{m1}	& a_{m2}	& ...	& a_{mn}  \\
	\end{pmatrix}.$$
\end{definition}

Позначають матриці і в такий спосіб: $A = (a_{ij})_{i=1, ..., m,\, j=1, ..., n}$, де $a_{ij}$ --- це елемент
матриці, що стоїть в $i$-му рядку та $j$-му стовпчику.

Нехай A – квадратна матриця другого порядку (тобто $2 \times 2$).

\begin{definition}
	\textbf{Визначник матриці (детермінант матриці) $A$ другого порядку} --- це число, яке знаходиться за формулою:
	$$\det A = |A| = \left| \begin{matrix}
		x_1 & y_1 \\
		x_2 & y_2 \\
	\end{matrix} \right| = x_1y_2 - x_2y_1$$
\end{definition}

\begin{definition}
	\textbf{Визначник матриці (детермінант матриці) $A$ третього порядку} ---  це число, яке знаходиться за формулою (правило ''зірочки''): 
	$$\det A = |A| = \left| \begin{matrix}
		x_1 & y_1 & z_1 \\
		x_2 & y_2 & z_2 \\
		x_3 & y_3 & z_3 \\
	\end{matrix} \right| = x_1y_2z_3 + x_3y_1z_2 + x_2y_3z_1 - x_3y_2z_1 - x_2y_1z_3 - x_1y_3z_2$$
\end{definition}

Ця формула нам вже відома, адже саме так ми шукали мішаний добуток
векторів $\overline{a}, \overline{b}, \overline{c}$, які в базисі
$\{\overline{i}, \overline{j}, \overline{k}\}$ мають координати
$\overline{a} = (x_1, y_1, z_1), \overline{b} = (x_2, y_2, z_2), \overline{c} = (x_3, y_3, z_3)$

\section{Властивості визначників}

Усі властивості будемо формулювати і доводити для визначників 3-го порядку.
Але, як ми побачимо пізніше, усі наведені властивості будуть
виконуватися і для визначників довільного порядку.

Нехай $A$ квадратна матриця третього порядку, тоді $\det A = \overline{a}\overline{b}\overline{c}$,
де $\overline{a} = (x_1, y_1, z_1), \overline{b} = (x_2, y_2, z_2), \overline{c} = (x_3, y_3, z_3)$.

\begin{definition}
	\textbf{Транспонована матриця} --- це матриця $A^T$, отримана шляхом транспонування (транспоновки) елементів матриці $A$, тобто стовпчики і рядки, міняються місцями: $A^T = (a_{ij}^T)$, де $a_{ij}^T = a_{ji}$.
\end{definition}

Нехай $A = \begin{pmatrix}
	a_{11} & a_{12} & a_{13} \\
	a_{21} & a_{22} & a_{23} \\
	a_{31} & a_{32} & a_{33} \\
\end{pmatrix}$.

\begin{definition}
	\textbf{Мінор $M_{ij}$, елемента $a_{ij}$} --- це визначник матриці, яка отримана з
	матриці A викреслюванням $i$-го рядка та $j$-го стовпчика. 
\end{definition}

Наприклад: $M_{23} = \left|\begin{matrix}
	a_{11} & a_{12} \\
	a_{31} & a_{32} \\
\end{matrix}\right| = a_{11}a_{32} - a_{12}a_{31}$.

\begin{definition}
	\textbf{Алгебраїчне доповнення елемента $a_{ij}$} --- це добуток $A_{ij} = (-1)^{i+j}M_{ij}$.
\end{definition}

\subsection*{Властивості визначників:}

1) При транспонуванні матриці значення її визначника не зміниться: 
$$\det A = \det A^T$$
\begin{proof}
	Доведення випливає безпосередньо з правила “зірочки”
	
	$\det A^T = \left|\begin{matrix}
		x_1 & x_2 & x_3 \\
		y_1 & y_2 & y_3 \\
		z_1 & z_2 & z_3 \\
	\end{matrix}\right|
	= x_1y_2z_3 + x_3y_1z_2 + x_2y_3z_1 - x_3y_2z_1 - x_2y_1z_3 - x_1y_3z_2 = \det A$

	Ця властивість урівнює в “правах” стовпчики і рядки. Тому далі всі властивості будемо формулювати для рядків.
\end{proof}

2) Знак визначника змінюється, якщо будь-які два рядки поміняти місцями:
$$D(\overline{a}, \overline{b}, \overline{c}) = -D(\overline{b}, \overline{a}, \overline{c}) = -D(\overline{a}, \overline{c}, \overline{b}) = -D(\overline{c}, \overline{b}, \overline{a})$$

3) Спільний множник можна винести з довільного рядка за визначник:
$$D(\alpha\overline{a}, \overline{b}, \overline{c}) = D(\overline{a}, \alpha\overline{b}, \overline{c}) = D(\overline{a}, \overline{b}, \alpha\overline{c}) = \alpha D(\overline{a}, \overline{b}, \overline{c})$$

4) $D(\overline{a} + \overline{a}', \overline{b}, \overline{c}) = D(\alpha\overline{a}, \overline{b}, \overline{c}) + D(\alpha\overline{a}', \overline{b}, \overline{c})$

5) Визначник, рядки якого пропорційні, дорівнює нулю: 
$$D(\overline{a}, \alpha\overline{a}, \overline{c}) = 0$$

6) Визначник, який має два однакових рядки, дорівнює нулю: 
$$D(\overline{a}, \overline{a}, \overline{c}) = 0$$
	
7) Визначник матриці з нульовим рядком дорівнює нулю: 
$$D(\overline{a}, \overline{0}, \overline{c}) = 0$$	
	
8) Визначник не змінюється, якщо до якогось його рядка додати лінійну
комбінацію інших рядків:
$$D(\overline{a}, \overline{b}, \overline{c} + \alpha\overline{a} + \beta\overline{b})
= D(\overline{a}, \overline{b}, \overline{c})$$
	
\begin{proof}
	$D(\overline{a}, \overline{b}, \overline{c} + \alpha\overline{a} + \beta\overline{b}) = D(\overline{a}, \overline{b}, \overline{c}) + D(\overline{a}, \overline{b}, \alpha\overline{a}) + D(\overline{a}, \overline{b}, \beta\overline{b}) = D(\overline{a}, \overline{b}, \overline{c})$
\end{proof}
	
9) Визначник матриці $A$ дорівнює сумі добутків елементів будь-якого рядка на їх
алгебраїчні доповнення. 
	
Наприклад: $\det A = a_{11}A_{11} + a_{12}A_{12} + a_{13}A_{13}$.

Будемо казати, що в цьому прикладі ми розклали визначник за першим рядком.

\begin{proof}
	$\sum\limits_{i=1}^3 a_{1i}A_{1i}
	= a_{11}\left|\begin{matrix}
		a_{22} & a_{23} \\
		a_{32} & a_{33} \\
	\end{matrix} \right|
	- a_{12}\left|\begin{matrix}
		a_{21} & a_{23} \\
		a_{31} & a_{33} \\
	\end{matrix} \right|
	+ a_{13}\left|\begin{matrix}
		a_{21} & a_{22} \\
		a_{31} & a_{32} \\
	\end{matrix} \right|
	= a_{11}(a_{22}a_{33} - a_{23}a_{32})
	- a_{12}(a_{21}a_{33} - a_{23}a_{31})
	+ a_{13}(a_{21}a_{32} - a_{22}a_{31})
	= a_{11}a_{22}a_{33} - a_{11}a_{23}a_{32} - a_{12}a_{21}a_{33}
	+ a_{12}a_{23}a_{31} + a_{13}a_{21}a_{32} - a_{11}a_{22}a_{31}
	= \det A$.
\end{proof}

\begin{problem}
	З’ясувати, чи можуть вектори $\overline{a} = (2, -1, 2), \overline{b} = (1, 2, -3), \overline{c} = (3, -4, 7)$ утворювати базис у просторі $E^3$?
\end{problem}
\begin{solution}
	Вектори $\overline{a}, \overline{b}, \overline{c}$ будуть утворювати базис, якщо вони
	некомпланарні. Тобто об’єм паралелепіпеда, на них побудованого, не повинен
	дорівнювати нулю. Знайдемо мішаний добуток даних векторів:

	$\overline{a}, \overline{b}, \overline{c}
	= d(\overline{a}, \overline{b}, \overline{c})
	= \left|\begin{matrix}
		2  & -1  &  2  \\
		1  &  2  & -3  \\
		3  & -4  &  7  \\
	\end{matrix}\right|
	= \left|\begin{matrix}
		0  & -1  &  0  \\
		5  &  2  &  1  \\
		-5 & -4  & -1  \\
	\end{matrix}\right|
	=-(-1)\left|\begin{matrix}
		5  &  1  \\
		-5 & -1  \\
	\end{matrix}\right|
	= 0$
	
	Для обчислення цього визначника ми спочатку 2-й
	стовпчик помножили на 2 і
	додали його до 1-го та 3-го стовпчиків, а потім скористались властивістю 9. Таким
	чином, об’єм паралелепіпеда, побудованого на векторах $\overline{a}, \overline{b}$ і $\overline{c}$ дорівнює нулю,
	тобто ці вектори лежать в одній площині і слугувати базисом не можуть.
\end{solution}

\section{Лінійна залежність векторів}

Розглянемо систему векторів $\overline{a}_1, \overline{a}_2, ..., \overline{a}_n$.

\begin{definition}
	\textbf{Лінійна комбінація векторів} --- це вираз $\alpha_1\overline{a}_1 + \alpha_2\overline{a}_1 + ... + \alpha_n\overline{a}_n$, де $\alpha_i \in \mathbb{R}$.
\end{definition}


\begin{definition}
	Лінійна комбінація векторів $\overline{a}_1, \overline{a}_2, ..., \overline{a}_n$ --- це
	\textbf{тривіальна лінійна комбінація}, якщо всі $\alpha_i = 0$,
	і це \textbf{нетривіальна лінійна комбінація}, в протилежному випадку.
\end{definition}

\begin{definition}
	\textbf{Лінійно залежні вектори} $\overline{a}_1, \overline{a}_2, ...,\overline{a}_n$ є такими ,якщо існують
	такі числа $\alpha_1, \alpha_2, ..., \alpha_n$
	що виконується рівність $\alpha_1\overline{a}_1 + \alpha_2\overline{a}_2 + ... + \alpha_n\overline{a}_n = \overline{0}$,
	причому $|\alpha_1| + |\alpha_2| + ... + |\alpha_n| \neq 0$.
\end{definition}

\begin{definition}
	\textbf{Лінійно незалежні вектори} --- це вектори $\overline{a}_1, \overline{a}_2, ..., \overline{a}_n$, такі, що їх лінійна
	комбінація дорівнює нулю лише за умови, коли всі $\alpha_i = 0$.
\end{definition}

Інакше кажучи, вектори $\overline{a}_1, \overline{a}_2, ..., \overline{a}_n$ є лінійно незалежними, якщо ніяка їх
нетривіальна лінійна комбінація не дорівнює нульовому вектору. 

\subsection*{Властивості:}

\begin{claim}
	Вектори $\overline{a}_1, \overline{a}_2, ..., \overline{a}_n$ – лінійно залежні тоді і тільки тоді, коли хоча б
	один з них лінійно виражається через інші. 
\end{claim}
\begin{proof}
	1) Нехай $\overline{a}_1, \overline{a}_2, ..., \overline{a}_n$ --- це лінійно залежні вектори. Тоді
    $\alpha_1\overline{a}_1 + ... + \alpha_i\overline{a}_i + ... + \alpha_n\overline{a}_n = \overline{0}$ та $\alpha_i \neq 0$ для деякого $i$ . Звідси випливає, що
	$\overline{a}_i = -\dfrac{1}{\alpha_i}(\alpha_1\overline{a}_1 + ... + \alpha_{i-1}\overline{a}_{i-1} + \alpha_{i+1}\overline{a}_{i+1} + ... + \alpha_n\overline{a}_n)$, тобто вектор $\overline{a}_i$ лінійно
	виражається через інші вектори системи.
	
	2) Нехай $\overline{a}_i = \beta_1\overline{a}_1 + ... + \beta{i-1}\overline{a}_{i-1} + \beta{i+1}\overline{a}_{i+1} + ... + \beta_n\overline{a}_n)$ для деякого $i$. Тоді
	$\beta_1\overline{a}_1 + ... + \beta_{i-1}\overline{a}_{i-1} + (-1)\overline{a}_{i} + \beta_{i+1}\overline{a}_{i+1} + ... + \beta_{n}\overline{a}_{n} = \overline{0}$, тобто отримано нульову
	лінійну комбінацію, в якій коефіцієнт при векторі i a є ненульовим. 
\end{proof}

\begin{claim}
	Якщо один з векторів $\overline{a}_1, \overline{a}_2, ..., \overline{a}_n$ є нульовим, то система цих векторів
	є лінійно залежною. 
\end{claim}
\begin{proof}
	Припустимо, що $\overline{a}_1 = \overline{0}$. Тоді очевидно, що
	$1\overline{a}_1 + 0 \overline{a}_2 + ... + 0 \overline{a}_n = \overline{0}$, тобто за означенням дана система векторів є лінійно
	залежною $(\alpha_1 = 1 \neq 0)$. 
\end{proof}

\begin{claim}
	Якщо серед векторів $\overline{a}_1, \overline{a}_2, ..., \overline{a}_n$ є два однакових, то система векторів є
	лінійно залежною.
\end{claim}
\begin{proof}
	Доведення є очевидним.
\end{proof}

\begin{claim}
	Якщо серед $n$ векторів існує $k$ лінійно залежних векторів, то і всі $n$
	векторів є лінійно залежними.
\end{claim}
\begin{proof}
	Розглянемо лінійну комбінацію даних $n$ векторів
	$\alpha_1\overline{a}_1 + ... + \alpha_{k}\overline{a}_{k} + 0\overline{a}_{k+1} + ... + 0\overline{a}_{n} = \overline{0}$,при умові, що
	$\alpha_1\overline{a}_1 + ... + \alpha_{k}\overline{a}_{k} = \overline{0}$ і константи $\alpha_1, \alpha_2, ..., \alpha_{n-1}$ не всі рівні нулю. Ця
	комбінація є нетривіальною, що і доводить потрібний факт.
\end{proof}

\begin{claim}
	Якщо вектор $\overline{a}$ лінійно виражається через лінійно незалежні вектори
	$\overline{a}_1, \overline{a}_2, ..., \overline{a}_n$, то таке представлення єдине. 
\end{claim}
\begin{proof}
	(від супротивного). Припустимо, що вектор $\overline{a}$ лінійно виражається
	через дані вектори не єдиним чином, тобто
	$\overline{a} = \alpha_{1}\overline{a}_{1} + \alpha_{2}\overline{a}_{2} + ... + \alpha_{n}\overline{a}_{n}$ і $\overline{a} = \beta_{1}\overline{a}_{1} + \beta_{2}\overline{a}_{2} + ... + \beta_{n}\overline{a}_{n}$.
	Віднявши другу рівність від першої, маємо:
	$\overline{0} = (\alpha_1 - \beta_1)\overline{a}_1 + (\alpha_2 - \beta_2)\overline{a}_2 + ... + (\alpha_n - \beta_n)\overline{a}_n$.
	Оскільки $\overline{a}_1, \overline{a}_2, ..., \overline{a}_n$ є лінійно незалежними, то $\alpha_1 = \beta_1, \alpha_2 = \beta_2, ..., \alpha_n = \beta_n$, що
	суперечить припущенню про неєдиність представлення вектора $\overline{a}$.
\end{proof}

\begin{claim}
	Довільна підсистема лінійно незалежних векторів $\overline{a}_1$, $\overline{a}_2$, $...$, $\overline{a}_n$ є лінійно
	незалежною.
\end{claim}
\begin{proof}
	Застосуємо метод від супротивного. Нехай існує підсистема
	$\overline{a}_1, \overline{a}_2, ..., \overline{a}_n, k<n$, лінійно залежних векторів. Це означає, що
	$\alpha_1\overline{a}_1 + \alpha_2\overline{a}_2 + ... + \alpha_{k}\overline{a}_{k} = \overline{0}$, причому $|\alpha_1| + |\alpha_2| + ... + |\alpha_k| \neq 0$. Додамо до обох
	частин рівності нуль-вектор: $0\overline{a}_{k+1} + ... + 0\overline{a}_{n}$. В результаті отримаємо:
	$\alpha_1\overline{a}_1 + \alpha_2\overline{a}_2 + ... + \alpha_k\overline{a}_k + 0\overline{a}_{k+1} + ... + 0\overline{a}_{n} = \overline{0}$ і $|\alpha_1| + |\alpha_2| + ... + |\alpha_k| \neq 0$.
	Це означає лінійну залежність векторів $\overline{a}_1, \overline{a}_2, ..., \overline{a}_n$, що протирічить припущенню.
	Отже, підсистема $\overline{a}_1, \overline{a}_2, ..., \overline{a}_n$, є лінійно незалежною. 
\end{proof}

\begin{claim}
	Якщо після доповнення системи $\overline{a}_1, \overline{a}_2, ..., \overline{a}_n$ лінійно незалежних
	векторів вектором $\overline{a}$, отримали лінійно залежну систему, то вектор $\overline{a}$ лінійно
	виражається через вектори $\overline{a}_1, \overline{a}_2, ..., \overline{a}_n$.
\end{claim}
\begin{proof}
	Оскільки вектори $\overline{a}_1, \overline{a}_2, ..., \overline{a}_n, \overline{a}$ є лінійно залежними, то існують
	такі константи $\alpha_1, \alpha_2, ..., \alpha_n, \alpha_{n+1}$, не всі рівні нулю, що
	
	$\alpha_1\overline{a}_1 + \alpha_2\overline{a}_2 + ... + \alpha_n\overline{a}_n + \alpha_{n+1}\overline{a} = \overline{0}$. (*)
	
	При цьому саме $\alpha_{n+1} \neq 0$. Доведемо цей факт методом від супротивного. Якщо
	$\alpha_{n+1} = 0$, то $\alpha_{n+1}\overline{a} = \overline{0}$ і $\alpha_1\overline{a}_1 + \alpha_2\overline{a}_2 + ... + \alpha_n\overline{a}_n = \overline{0}$, причому серед чисел
	$\alpha_1, \alpha_2, ..., \alpha_n$ існують ненульові. Але в цьому випадку вектори $\overline{a}_1, \overline{a}_2, ..., \overline{a}_n$ є 
	лінійно залежними, що суперечить умові твердження. Отже, $\alpha_{n+1} \neq 0$, тому з
	рівності (*) випливає, що $\overline{a} = \left(-\dfrac{\alpha_1}{\alpha_{n+1}}\right)\overline{a}_1 + ... + \left(-\dfrac{\alpha_n}{\alpha_{n+1}}\right)\overline{a}_n$.
\end{proof}

\section*{Геометричний сенс лінійної залежності}

\begin{claim}
	Два вектори $\overline{a}$ і $\overline{b}$ є лінійно залежними тоді і тільки тоді, коли $\overline{a} \parallel \overline{b}$.
\end{claim}

\begin{claim}
	Три вектори $\overline{a}, \overline{b}, \overline{c}$ є лінійно залежними тоді і тільки тоді, коли $\overline{a}, \overline{b}, \overline{c}$ є компланарними. 
\end{claim}

\begin{claim}
	Довільні чотири вектори $\overline{a}, \overline{b}, \overline{c}, \overline{d} \in E^3$ є завжди лінійно залежними.
\end{claim}






