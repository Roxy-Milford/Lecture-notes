\chapter{Аналітична геометрія у просторі}


%\section{Площина у просторі}
%\section{Пряма у просторі}
%\section{Поверхні другого порядку}


\section{Площина у просторі}

Нехай у просторі задана площина $P$, точка $M_0(x_0,y_0,z_0)$ та
$n = (A,B,C) \neq 0$, $\overline{n} \perp P$.

Озн. Вектор $n \perp P$ називається нормальним вектором площини,
задається він єдиним чином з точністю до ненульової константи.







Нехай т. $M(x,y,z)$ -- довільна точка простору.
$\overline{M_0 M} = (x-x_0,y-y_0,z-z_0)$.

Тв. $M \in P \Leftrightarrow \overline{n} \perp \overline{M_0 M} \Rightarrow (\overline{n},\overline{M_0 M}) = 0$


$$A(x - x_0) + b(y - y_0) + C(z - z_0) = 0.$$

Отримали рівняння площини $P$, яка задана
точкою та нормальним вектором. Розкриваючи
дужки, маємо:

$$Ax + By + Cz - Ax_0 - By_0 - Cz_0 = 0;$$

$Ax + By + Cz + D = 0$ --- загальне рівняння
площини.


\section{Неповні рівняння площини}

Виходячи з геометричного тлумачення коефіцієнтів $A$, $B$, і $C$ як
координат вектора її нормалі, дослідимо питання про їх вплив на розміщення
площини відносно системи координат.

$P: Ax + Bx + Cz + D = 0$; $M_0(x_0,y_0,z_0) \in P$;

1) $\overline{n} = (0,0,C) \parallel (0,0,1)$, $n \perp XOY$,
$P : z - z_0 = 0 \Rightarrow z = z_0$ --- площина $\perp XOY$.

Частковий випадок: $z = 0$ --- сама площина $XOY$.

2) $\overline{n} = (0,B,0) \parallel (0,1,0)$; $P: y = y_0 \parallel XOZ$.

3) $\overline{n} = (A,0,0) \parallel (1,0,0)$; $P: x = x_0 \parallel YOZ$.

4) $ABCD \neq 0 \Rightarrow (0,0,0) \notin P$ і $P$ не паралельна жодній координатній
площині. Така площина називається площиною загального розташування.


\section{Рівняння площини «у відрізках»}

Нехай $P: Ax + By + Cz + D = 0$; $ABCD \neq 0 \Rightarrow$

$$\dfrac{x}{\dfrac{D}{A}} + \dfrac{y}{\dfrac{D}{B}} + \dfrac{z}{\dfrac{D}{C}} = 1;
\dfrac{x}{a} + \dfrac{y}{b} + \dfrac{z}{c} = 1
\text{ --- рівняння площини у відрізках}.$$


Зауважимо, що числа $a$, $b$, $c$ мають простий геометричний зміст: вони
дорівнюють величинам відрізків, які відтинає площина на осях $Ox$, $Oy$, і
$Oz$ (з урахуванням знаків).

Справді, якщо дану площину $P$
перетнути площинами $y = 0$ і $z = 0$,
маємо, що точка $M_1(a,0,0) \in P$.
Аналогічно $M_2(0,b,0) \in P$,
$M_3(0,0,c) \in P$.


\section{Нормальне рівняння площини}

Нехай площина $P$, що не
проходить через $O(0,0,0)$, задана так:
її нормальний вектор $\overline{n}$ проведено з
початку координат в напрямку
площини; $|\overline{n}| = p > 0$; його кути з
координатними осями --- ($\alpha$, $\beta$, $\gamma$).

$$\overline{n}_0 = \dfrac{1}{|\overline{n}|}\overline{n}; \overline{n}_0 = {\cos \alpha, \cos \beta, \cos\gamma};$$

$$|\overline{n}_0| = 1.$$

Точка $M(x,y,z)$ --- довільна; $\overline{OM} = (x, y,z)$.

Зрозуміло, що $M \in P \Leftrightarrow \text{np}_{\overline{n}}\overline{OM} = p$; $\text{np}_{\overline{n}}\overline{OM} = \text{np}_{\overline{n}_0} \overline{OM}(\overline{n}_0, \overline{OM}) = p$ або
$x \cos \alpha + y \cos \beta + z \cos \gamma = p$;

$x \cos \alpha + y \cos \beta + z \cos \gamma - p = 0$ --- нормальне рівняння площини.

Щоб звести рівняння $Ax + By + Cz + D$ до нормального, треба праву і
ліву частину помножити на $M = \dfrac{- \text{sigh} D }{\sqrt{A^2 + B^2 + C^2}}$.

Приклад. Звести рівняння площини $P: 2x - 3y + 6z + 14 = 0$ до нормального
виду.

Розв’язання. Нормуючим множником для даного рівняння буде
$M = \dfrac{- 1 }{\sqrt{2^2 + 3^2 + 6^2}} = -\dfrac{1}{7}$. Знак числа $M$ завжди протилежний знаку вільного
члена D.

$$P: -\dfrac{1}{7}(2x - 3y + 6z + 14) = 0; p = 2.$$


\section{Відхилення і відстань від точки до площини}

Нехай задано площину $P: Ax + By + Cz + D = 0$; $D \neq 0$ і точку
$M_0(x_0,y_0,z_0) \notin P$.

$P: x \cos \alpha + y \cos \beta + z cos \gamma - p = 0$.

І нехай $\rho(M_0,P) = d > 0$ --- відстань від $M_0$ до $P$.

Озн. Відхиленням $M_0$ від $P$ (позначається $\delta_{M_0P}$) називається величина

$\delta_{M_0P} = \left\{\begin{matrix}
	d \text{, якщо} M_0 \text{ і } O(0,0,0)\text{ лежать по різні боки від } P, \\
	-d \text{, якщо} M_0 \text{ і } O(0,0,0)\text{ лежать по одну сторону від } P. \\
\end{matrix} \right.$



Побудуємо $P' \parallel P$, $M_0 \in  P'$.

$P': x \cos \alpha + y \cos \beta + z \cos \gamma - (p + \delta_{M_0P}) = 0$.

Враховуючи, що $M_0 \in P$, маємо:

$$x_0 \cos \alpha + y_0 \cos \beta + z_0 \cos \gamma - p - \delta_{M_0P} = 0 \Rightarrow$$.

$$\delta_{M_0P} = x_0 \cos \alpha + y_0 \cos \beta + z_0 \cos \gamma - p;$$

$$d = \rho(M_0,P) = |\delta_{M_0P}| = |x_0 \cos \alpha + y_0 \cos \beta + z_0 \cos \gamma - p|.$$



\section{Взаємне розташування площин}

$$P_1: A_1 x + B_1 y + C_1 z + D_1 = 0, \overline{n}_1 = (A_1,B_1,C_1), M_1(x_1,y_1,z_1)\in P_1.$$

$$P_2: A_2 x + B_2 y + C_2 z + D_2 = 0, \overline{n}_2 = (A_2,B_2,C_2), M_2(x_2,y_2,z_2)\in P_2.$$

1) Площини $P_1$ і $P_2$ співпадають; $P_1 = P_2$.

Тоді $\rang \begin{pmatrix}
	A_1 & B_1 & C_1 & D_1 \\
	A_2 & B_2 & C_2 & D_2 \\
\end{pmatrix} = 1.$

2) $P_1 \neq P_2$, $P_1 \parallel P_2$.

Тоді $\overline{n}_1 \parallel \overline{n}_2$ і $\rang \begin{pmatrix}
	A_1 & B_1 & C_1 \\
	A_2 & B_2 & C_2 \\
\end{pmatrix} = 1$, а $\rang \begin{pmatrix}
	A_1 & B_1 & C_1 & D_1 \\
	A_2 & B_2 & C_2 & D_2 \\
\end{pmatrix} = 2$.

Відстань від $P_1$ до $P_2$ --- $\rho(P_1,P_2)$ доцільно шукати, як відстань від $M_2$ до $P_1$:

$$\rho(P_1,P_2) = \rho(M_2,P_1) = \dfrac{|A_1 x_2 + B_1 y_2 + D_1 z_2 + D_1|}{\sqrt{A_1^2 + B_1^2 + C_1^2}}.$$


3) $P1 \nparallel P_2$;

$$\overline{n}_1 \nparallel \overline{n}_2 \Leftrightarrow \rang \begin{pmatrix}
	A_1 & B_1 & C_1 \\
	A_2 & B_2 & C_2 \\
\end{pmatrix} = 2.$$


Якщо $\alpha = (\widehat{P_1,P_2})$, то $\cos \alpha = |\cos(\widehat{\overline{n}_1,\overline{n}_2})|
= \dfrac{|A_1A_1 + B_1B_1 + C_1C_1|}{\sqrt{A_1^2 + B_1^2 + C_1^2}\sqrt{A_2^2 + B_2^2 + C_2^2}}$.


\section{Пряма у просторі}


Пряма у просторі задається двома принципово різними способами.

I спосіб. Нехай $l$ --- пряма, точка $M_0(x_0,y_0,z_0) \in l$,


Озн. Вектор $\overline{q} = (a,b,c) \neq 0$, $\overline{q} \parallel l$ називається напрямним вектором
прямої $l$.

Точка $M(x,y,z)$ --- довільна точка простору.


$\overline{M_0M} = (x-x_0,y-y_0,z-z_0)$.
Зрозуміло, що $M \in L \Leftrightarrow \overline{M_0M} \parallel q$ або

$$\dfrac{x-x_0}{a} = \dfrac{y-y_0}{b} = \dfrac{z-z_0}{c}.(1)$$

Ми отримали так зване канонічне
рівняння прямої.


Рівняння $\dfrac{x-x_0}{a} = \dfrac{y-y_0}{b}$, $\dfrac{x-x_0}{a} = \dfrac{z-z_0}{c}$
і $\dfrac{y-y_0}{b} = \dfrac{z-z_0}{c}$ --- рівняння
площин, що проектують пряму $l$ на координатні площини.

II спосіб. Площини $P_1$ і $P_2$, $P_1 \nparallel P_2$ задаються рівняннями:

$$P1: A_1x + B_1y + C_1z + D_1 = 0.$$
$$P2: A_2x + B_2y + C_2z + D_2 = 0.$$

Пряму $l$ задають як перетин $P_1$ і $P_2$:

$$l: \left\{ \begin{matrix}
	A_1x + B_1y + C_1z + D_1 = 0
	A_2x + B_2y + C_2z + D_2 = 0
\end{matrix} \right. (2)$$


Для того, щоби звести рівняння прямої типу (2) до вигляду (1), треба
знайти напрямний вектор $\overline{q}$ і якусь точку $M_0(x_0,y_0,z_0)$, що належить $l$.

$\overline{q} \perp \overline{n}_1(A_1,B_1,C_1)$, $\overline{q} \perp \overline{n}_2 = (A_2,B_2,C_2) \Rightarrow \overline{q} = [\overline{n}_1,\overline{n}_2].$


Якщо систему (2) розглядати як неоднорідну систему лінійних рівнянь
(за умовою вона сумісна!), то якийсь її частковий розв’язок $(x_0,y_0,z_0)$ дасть
нам координати точки $M_0$.

Як наслідок рівняння прямої типу (1) є її параметричне рівняння:

$$\dfrac{x-x_0}{a} = \dfrac{y-y_0}{b} = \dfrac{z-z_0}{c} = t, t \in (-\infty;+\infty) \Rightarrow 
\left\{ \begin{matrix}
	x = at +x_0; \\
	y = bt +y_0; \\
	z = ct +z_0. \\
\end{matrix} \right.$$


\section{Взаємне розташування прямої і площини}


Нехай дано:
пряма $l: \dfrac{x-x_0}{a} = \dfrac{y-y_0}{b} = \dfrac{z-z_0}{c}$, $\overline{q} = (a,b,c)$,
$M_0(x_0,y_0,z_0)$,
площина $P: Ax + By + Cz + D = 0$, $\overline{n} = (A,B,C)$.

1. $l \subset P$ --- пряма лежить в площині. 
$n \perp q$,
$M_0 \in P$, $(\overline{n},\overline{q}) = 0 \Rightarrow Aa + Bb + Cc = 0$.


2. $l \parallel P$ --- пряма паралельна площині.

$\overline{q} \perp \overline{n}$, $M_0 \notin P$.

Відстань між прямою і площиною
$\rho(l,P) = \rho(M_0,P)$ і
знаходиться стандартним методом.


3. $l \nparallel P$, $l \cap P = \{K\}$ — пряма перетинає площину в точці K.

Вектори $\overline{n}$ та $\overline{q}$ --- не перпендикулярні
$\Rightarrow (\overline{n},\overline{q}) \neq 0 \Rightarrow Aa + Bb + Cc \neq 0$.

Зауваження. Щоб знайти координати точки $K$, доцільно
використовувати параметричне рівняння прямої.

Частковий випадок:
$l \perp P \Rightarrow \overline{q} \parallel \overline{n}$.


\section{Взаємне розташування двох прямих}

$$l_1: \dfrac{x-x'_0}{a_1} = \dfrac{y-y'_0}{b_1} = \dfrac{z-z'_0}{c_1},
\overline{q}_1 = (a_1,b_1,c_1),
M_1(x'_0,y'_0,z'_0) \in l_1;$$

$$l_2: \dfrac{x-x''_0}{a_2} = \dfrac{y-y''_0}{b_2} = \dfrac{z-z''_0}{c_2},
\overline{q}_2 = (a_2,b_2,c_2),
M_2(x''_0,y''_0,z''_0) \in l_2.$$

1. $l_1 \parallel l_2$, ($l_1 \neq l_2$) --- прямі не співпадають і паралельні.


У цьому випадку $\overline{q}_1 \parallel \overline{q}_2$, а $M_1 \notin l_2$.

Відстань між паралельними прямими
простіше за все шукати, як висоту
паралелограма, побудованого на
векторах $\overline{q}_1$ і $\overline{M_1M_2}$:

$$\rho(l_1,l_2) = h = \dfrac{|[\overline{q}_1,\overline{M_1,M_2}]|}{|\overline{q}_1|}.$$



2. $l_1 \cap l_2 = \{K\}$ --- прямі перетинаються в точці $K$.


У цьому випадку вектори $q_1$, $q_2$ і
$\overline{M_1M_2}$ --- компланарні,
тобто $\overline{q}_1 \overline{q}_2 \overline{M_1M_2} = 0 \Rightarrow$

$$\Rightarrow \left| \begin{matrix}
	a_1 & b_1 & c_1 \\
	a_2 & b_2 & c_2 \\
	x'_0 - x''_0 & y'_0 - y''_0 & z'_0 - z''_0 \\
\end{matrix} \right| = 0.$$



3. $l_1 \nparallel l_2$, $l_1 \cap l_2 = \varnothing$ --- прямі мимобіжні.


У такому випадку прямі не
паралельні і не лежать в одній
площині $\Rightarrow$

$V = |\overline{q}_1 \overline{q}_2 \overline{M_1M_2}| \neq 0$, де $V$ --- об’єм
паралелепіпеда, побудованого на цих
трьох векторах.

Довести, що відстань між прямими

$$\rho(l_1,l) = \dfrac{|\overline{q}_1 \overline{q}_2 \overline{M_1M_2}|}{|[\overline{q}_1,\overline{q}_2]|}$$


Спільний перпендикуляр $L$ до прямих $l_1$ і $l_2$ пропонується знаходити за
такою схемою:

а) Знайти нормальні вектори площин $P_1$ і $P_2$, де $P_1 \supset l_1$, $P_1 \parallel l_2 \Rightarrow$

$\Rightarrow \overline{n}_1 \perp \overline{q}_1$, $\overline{n}_1 \perp \overline{q}_2 \Rightarrow \overline{n}_1 = [\overline{q}_1,\overline{q}_2]$, $\overline{n}_2 = \overline{n}_1$.

б). Знайти рівняння площин $P_3$ і $P_4$:

$$P_3 \supset l_1, P_3 \perp P_1 \Rightarrow \overline{n}_3 = [\overline{q}_1,\overline{n}_1], M_1 \in P_3,$$

$$P_4 \supset l_2, P_4 \perp P_1 \Rightarrow \overline{n}_4 = [\overline{q}_2,\overline{n}_1], M_2 \in P_4.$$


в). $L: \left\{ \begin{matrix}
	P_3 \\
	P_4 \\
\end{matrix} \right. .$


\section{Поверхні другого порядку}

Загальний вид поверхні другого порядку:

$$Ax^2 + By^2 + Cz^2 + Dxy + Fxz + Eyz + Ix + Gy + Kz + L = 0.$$

При різних значеннях коефіцієнтів це можуть бути власне поверхні,
площини, прямі, точки, уявні поверхні. Ми ж будемо розглядати канонічні
рівняння поверхонь другого порядку. Основний метод їх дослідження ---
метод перетину.


\section{Еліпсоїд}

$$\dfrac{x^2}{a^2} + \dfrac{y^2}{b^2} + \dfrac{z^2}{c^2} = 1 (\text{Тут і далі } a  > 0, b > 0 , c > 0).$$

$$\dfrac{x^2}{a^2} = 1 - \dfrac{y^2}{b^2} - \dfrac{z^2}{c^2} \leqslant 1 \Rightarrow |x| \leqslant a.$$


а це означає, що вся наша поверхня лежить між площинами $x = - a$ і
$x = a$. З аналогічних міркувань вона розташована між $y = -b$, $y = b$ та $z = -c$,
$z = c$. Перетнемо цю поверхню площиною $z = C$, $|C| \leqslant c$.

$$\dfrac{x^2}{a^2} + \dfrac{y^2}{b^2} = 1 - \dfrac{C^2}{c^2}$$ --- проекція перетину на $XOY$.

$\dfrac{x^2}{a^2\left(1-\dfrac{C^2}{c^2}\right)} + \dfrac{y^2}{b^2\left(1-\dfrac{C^2}{c^2}\right)} = 1$ ---
отримали еліпс з півосями $a\sqrt{1 - \dfrac{C^2}{c^2}}$ та $b\sqrt{1 - \dfrac{C^2}{c^2}}$.


Зрозуміло, що з ростом $C$ півосі зменшуються. Якщо $z = 0$, то в
перетині маємо еліпс $\dfrac{x^2}{a^2} + \dfrac{y^2}{b^2} = 1$.


Перетинаючи нашу поверхню двома іншими координатними площинами
$y = 0$ та $z = 0$, маємо ще два еліпси: $\dfrac{x^2}{a^2} + \dfrac{z^2}{c^2} = 1$
і $\dfrac{y^2}{b^2} + \dfrac{z^2}{c^2} = 1$.

Отриманої інформації вистачає,
щоб побудувати поверхню.


\section{Гіперболоїд}

а) однопорожнинний

$$\dfrac{x^2}{a^2} + \dfrac{y^2}{b^2} - \dfrac{z^2}{c^2} = 1.$$

Спочатку дещо з’ясуємо про розташування цієї поверхні.

$$\dfrac{x^2}{a^2} + \dfrac{y^2}{b^2} = 1 + \dfrac{z^2}{c^2} \geqslant 1 \Rightarrow \dfrac{x^2}{a^2} + \dfrac{y^2}{b^2} \geqslant 1.$$

Це означає, що всі точки поверхні проектуються поза еліпсом

$$\dfrac{x^2}{a^2} + \dfrac{y^2}{b^2} = 1.$$

Перетнемо поверхню площиною $z = C$. 

$$\dfrac{x^2}{a^2} + \dfrac{y^2}{b^2} = 1 + \dfrac{C^2}{c^2} \Rightarrow 
\dfrac{x^2}{a^2\left(1+\dfrac{C^2}{c^2}\right)} + \dfrac{y^2}{b^2\left(1+\dfrac{C^2}{c^2}\right)} = 1.$$


Тобто, лінії перетину проектуються у еліпси, півосі яких зростають зі
збільшенням $C$.

Якщо поверхню перетнути площинами $x = 0$ і $y = 0$, отримаємо у
вертикальних координатах площинах гіперболи: 

$\dfrac{y^2}{b^2} - \dfrac{x^2}{c^2} = 1$, $\dfrac{x^2}{a^2} - \dfrac{x^2}{c^2} = 1$.


Цієї інформації вистачає, щоб побудувати поверхню.


б) двопорожнинний

$$\dfrac{x^2}{a^2} + \dfrac{y^2}{b^2} - \dfrac{z^2}{c^2} = -1.$$

$$\dfrac{z^2}{c^2} = \dfrac{x^2}{a^2} + \dfrac{y^2}{b^2} + 1 \geqslant 1, \Rightarrow
\dfrac{z^2}{c^2} \geqslant 1 \Rightarrow |z| \geqslant c.$$

Це означає, що вся поверхня лежить вище ніж площина $z = c$ і нижче
ніж площина $z = -c$. В перетині з площинами $z = C$ маємо еліпси, параметри 
яких збільшуються з ростом $|c|$, а в перетині з площинами $y = 0$ і $x = 0$ ---
гіперболи

$\dfrac{x^2}{a^2} - \dfrac{z^2}{c^2} = -1$ та $\dfrac{y^2}{b^2} - \dfrac{z^2}{c^2} = -1$.


\section{Параболоїд}

а) еліптичний

$$z = \dfrac{x^2}{a^2} + \dfrac{y^2}{b^2}.$$

Уся поверхня лежить у верхній частині простору, бо $z \geqslant 0$.

У перетині з площинами $z = C \geqslant 0$ маємо еліпси, що збільшуються з
ростом $C$.

А в перетині з координатними площинами $y = 0$ і $x = 0$ --- параболи

$z = \dfrac{x^2}{a^2}$ і $z = \dfrac{y^2}{b^2}$.



б) гіперболічний

$$z = \dfrac{x^2}{a^2} - \dfrac{y^2}{b^2}.$$

У перетині з площиною $y = 0$ маємо параболу $z = \dfrac{x^2}{a^2}$, а з площиною
$x = 0$ --- $z = \dfrac{y^2}{b^2}$. При $z = C > 0$ у перетині маємо гіперболи
$\dfrac{x^2}{a^2 c} - \dfrac{y^2}{b^2 C} = 1$, а при $z = c < 0$ --- спряжені гіперболи
$\dfrac{x^2}{a^2 c} - \dfrac{y^2}{b^2 C} = 1$.


\section{Конус другого порядку}

$$\dfrac{x^2}{a^2} + \dfrac{y^2}{b^2} - \dfrac{z^2}{c^2} = 0.$$

Ця поверхня у перетині з площинами $z = C$ утворюють еліпси.

А при $y = 0$ маємо дві прямі:$\dfrac{x^2}{a^2} = \dfrac{z^2}{c^2} \Rightarrow
z = \dfrac{c}{a}x$ і $z = - \dfrac{c}{a}x$

При $x = 0$ --- прямі $z = \pm \dfrac{c}{b}y$

\section{Циліндри другого порядку}

а) еліптичний

$$\dfrac{x^2}{a^2} + \dfrac{y^2}{b^2} = 1.$$


б) гіперболічний

$$\dfrac{x^2}{a^2} - \dfrac{y^2}{b^2} = 1.$$


в) параболічний

$$y^2 = 2px$$


