\chapter{Многочлени}


\section{Алгебра многочленів}

\begin{definition}
	Многочленом (або поліномом) степені $n$ $(n \geqslant 0, n \in \mathbb{Z})$ від змінної $x$
називається вираз вигляду $f(x) = a_0x^n + a_1x^{n-1} + ... + a_{n-1}x + a_n = \sum\limits_{k = 0}^n a_kx^{n-k}$, де
$a_i \in \mathbb{C}, i = 0, ..., n$, причому $a_0 \neq 0$.
\end{definition}

Многочлен нульової степені --- це просто константи. Вводиться нульовий
многочлен: $0(x) = 0x^n + 0x^{n-1} + ... + 0x + 0$. Множину многочленів
степені $\leqslant n$ будемо позначати $P_n$.

\section{Операції над многочленами}

Нехай $g(x) \in P_n$: $g(x) = b_0x^n + b_1x^{n-1} + ... + b_{n-1}x + b_n$. Два
многочлени $f(x)$ та $g(x)$ вважаються рівними, якщо в них рівні коефіцієнти при
однакових степенях, тобто, $a_i = b_i$, $i = 0, ..., n$.


1) Множення многочлена на константу $\alpha \in \mathbb{C}$:

$$(\alpha f)(x) = \alpha\sum\limits_{k=0}^n a_k x^{n-k} = \sum\limits_{k=0}^n (\alpha a_k) x^{n-k}.$$

2) Додавання многочленів $f(x)$ та $g(x)$:

$$(f + g)(x) = \alpha\sum\limits_{k=0}^n (a_k+b_k)x^{n-k}.$$

Нехай $f, g, p \in P_n$, константи $\alpha, \beta \in \mathbb{C}$. Операції множення та додавання
мають наступні властивості.
 
\begin{enumerate}
	\item $f + g = g + f$.
	\item $f + (g + p) = (f + g) + p$.
	\item $(\alpha\beta)f = \alpha(\beta f)$.
	\item $(\alpha + \beta)f = \alpha f + \beta f$.
	\item $\alpha(f + g) = \alpha f + \alpha g$.
	\item Існує нульовий многочлен $0(x) \in P_n$: $f(x) + 0(x) = f(x)$.
	\item Для будь-якого многочлена $f(x)$ існує протилежний многочлен --- $f(x)$, такий що $f(x) + (-f)(x) = 0(x)$.
	\item $1 \cdot f(x0 = f(x)$; $(-f)(x) = (-1)f(x)$.
\end{enumerate}

3) Множення многочленів $f(x)$ та $g(x)$: щоб помножити многочлен $f(x)$ на
многочлен $g(x)$, потрібно кожний член многочлена $f(x)$ помножити на кожний
член $g(x)$, додати отримані добутки і звести подібні члени. Степінь добутку двох
многочленів дорівнює завжди сумі степенів множників.

4) Ділення многочленів $f(x)$ на $g(x)$ ($g(x) \neq 0$) з остачею. Позначимо
$\deg g(x)$ --- степінь многочлена $g(x)$.

\begin{theorem}
	Для двох довільних многочленів $f(x)$ та $g(x)$ знайдуться
	многочлени $p(x)$ і $r(x)$ такі, що
	
	$$f(x) = g(x)p(x) + r(x)$$

	де $\deg r(x) < \deg g(x)$ і таке представлення єдине.
	
	Многочлен $p(x)$ називається часткою, а $r(x)$ --- остачею від ділення
	многочлена $f(x)$ на $g(x)$.
\end{theorem}
\begin{proof}
	Нехай $f(x) = a_0x^n + a_1x^{n-1} + ... + a_{n-1}x + a_n$, $\deg f(x) = n$,
	$g(x) = b_0x^m + b_1x^{m-1} + ... + a_m$, $\deg g(x) = m$. Розглянемо два випадки:
	
	1) $m > n: f(x) = g(x) 0(x) + f(x)$, тобто частка --- нульовий многочлен, а остача
	--- це $f(x)$; 
	
	2) $n \geqslant m$: ділимо “кутом” многочлен $f(x)$ на $g(x)$ (відомо зі шкільного курсу
	алгебри); тоді $f(x) = g(x)p(x) + r(x)$, де $p(x)$ --- частка та $r(x)$ --- остача від
	ділення.
	
	Єдиність представлення доводимо від супротивного. Припустимо, що існує інший
	розклад, відмінний від даного: $f(x) = g(x)\tilde{p}(x) + \tilde{r}(x)$. Віднімемо ці розклади. Тоді

	$$0 = g(x)(p(x) -\tilde{p}(x)) + r(x) - \tilde{r}(x),$$\\
		
	$$g(x)(p(x) -\tilde{p}(x)) = \tilde{r}(x) - r(x). (**)$$
	
	З останньої рівності випливає, що
	
	$$\deg g(x) + \deg(p(x) -\tilde{p}(x)) = \deg(\tilde{r}(x) - r(x)).$$

	Оскільки степінь многочлена $\tilde{r}(x) - r(x)$ менша степені $g(x)$, то рівність $(**)$
	можлива лише тоді, коли $p(x) -\tilde{p}(x) = 0$. Це означає, що $p(x) = \tilde{p}(x)$ та $r(x) = \tilde{r}(x)$,
	що доводить єдиність розкладу (*).
\end{proof}

\begin{definition}
	Якщо $r(x) = 0$, то кажуть, що $f(x)$ ділиться на $g(x)$ без остачі (націло)
	і позначають: $f(x) \divby g(x)$.
\end{definition}

\begin{claim}
	$f(x) \divby g(x) \Rightarrow \alpha f(x) \divby g(x)$, де $\alpha = const$.
\end{claim}

\begin{claim}
	$f(x) \divby g(x)$ та $q(x) \divby g(x) \Rightarrow \alpha f(x) + \beta q(x) \divby g(x)$,
	де $\alpha, \beta = const$. 
\end{claim}

\section{Найбільший спільний дільник двох многочленів}

\begin{definition}
	Якщо кожен з многочленів $f(x)$ та $g(x)$ ділиться без остачі на
	многочлен $\varphi(x)$, то $\varphi(x)$ називається спільним дільником $f(x)$ і $g(x)$.
\end{definition}

\begin{definition}
	Найбільшим спільним дільником (НСД) многочленів $f(x)$ та $g(x)$
	називається такий їх спільний дільник $d(x)$, який ділиться на всі інші спільні
	дільники многочленів і позначається $(f(x),g(x)) = d(x)$.
\end{definition}

Для будь-яких двох многочленів, які не дорівнюють одночасно нулю, НСД
існує і визначається однозначно з точністю до множників-констант. Із усіх НСД
вибирається той, у якого старший коефіцієнт дорівнює 1.

\begin{definition}
	Два многочлени $f(x)$ та $g(x)$ називаються взаємно простими, якщо $(f,g) = 1$.
\end{definition}

Якщо відомі розклади многочленів $f(x)$ та $g(x)$ на лінійні множники, то НСД
$(f,g)$ легко знаходяться. Справді, нехай

$$f(x) = a_0(x-\alpha_{1})^{k_1}...(x-\alpha_{p})^{k_p}(x-\beta_{1})^{n_1}...(x-\beta_{l})^{n_l},$$

$$f(x) = a_0(x-\alpha_{1})^{i_1}...(x-\alpha_{p})^{i_p}(x-\gamma_{1})^{m_1}...(x-\gamma_{s})^{m_s},$$

де числа $\alpha_j, \beta_j, \gamma_j$ є попарно різними. Тоді

$$(f,g) = (x-\alpha_1)^{\min(k_1,i_1)}...(x-\alpha_p)^{\min(k_p,i_p)}$$

Якщо ж розклад многочленів на множники невідомий, то для знаходження
НСД використовують алгоритм Евкліда. 

\section{Алгоритм Евкліда}

Нехай $\deg f(x) \geqslant \deg g(x)$. За теоремою про ділення многочленів з остачею:

$$f(x) = g(x)p_1(x) + r_1(x), \deg r_1(x) < \deg g(x);$$

$$g(x) = r_1(x)p_2(x) + r_2(x), \deg r_2(x) < \deg r_1(x);$$

$$r_1(x) = r_2(x)p_3(x) + r_3(x), \deg r_3(x) < \deg r_2(x);$$

$$...$$

$$r_{k-2}(x) = r_{k-1}(x)p_k(x) + r_k(x), \deg r_k(x) < \deg r_{k-1}(x);$$

$$r_{k-1}(x) = r_{k}(x)p_{k+1}(x) + r_{k+1}(x), \deg r_{k+1}(x) < \deg r_{k}(x);$$

$$r_{k}(x) = r_{k+1}(x)p_{k+2}(x);$$

$$d(x) = r_{k+1}(x).$$

Остання ненульова остача $r_{k+1}(x)$ є НСД многочленів $f(x)$ і $g(x)$.

\begin{problem}
	Знайти НСД многочленів

	$$f(x) = x^4 + 3x^3 - x^2 - 4x - 3, g(x) = 3x^3 + 10x^2 + 2x -3.$$
\end{problem}

\begin{solution}
	\textit{Розв’язання задачі проводимо за наступною схемою}:

	\begin{enumerate}
		\item $f:g$ і знаходимо остачу $r_1$; 
		\item $g:r_1$ і знаходимо остачу $r_2$;
		\item $r_1:r_2$ і знаходимо остачу $r_3$;
		
		\indent \vdots

		\item $r_k:r_{k+1}$ (остача 0).
	\end{enumerate}

	Ділимо дані многочлени “кутом”, при цьому можна множити ділене та дільник
	на будь-які ненульові числа (навіть у процесі самого ділення). У нашому випадку
	ділимо $3f$ на $g$:

	$$x^4 + 3x^3 - x^2 -4x -3 = (3x^3 + 10x^2 + 2x - 3)(x - \dfrac{1}{3}) + (-\dfrac{5}{3}x^2 - \dfrac{25}{3}x - 10)$$

	Як $r_1(x)$ обираємо тричлен $x^2 + 5x +6$ (остачу помножили на $-\frac{3}{5}$). Далі

	$$3x^3 + 10x^2 + 2x - 3 = (x^2 + 5x + 6)(3x - 5) + (9x + 27)$$

	Як $r_2(x)$ обираємо двочлен $x+3$ (остачу помножили на $\frac{1}{9}$). Далі

	$$x^2 + 5x + 6 = (x + 3)(x + 2) + 0$$

	Шуканим НСД многочленів $f(x)$ та $g(x)$ є остання ненульова остача $r_2(x)$.
	Маємо $(f(x),g(x)) = x + 3$.
\end{solution}

\section{Корені многочленів}

\begin{definition}
	Число $x_0$ називається коренем многочлена $f(x)$, якщо $f(x_0) = 0$.
\end{definition}

\begin{definition}
	Корінь многочлена $x_0$ має кратність $k$, якщо $f(x) = (x-x_0)^k\varphi(x)$,
	причому $\varphi(x) \neq 0$ і $f(x)$ не ділиться на $(x-x_0)^{k+1}$. Корені кратності $k = 1$
	називаються простими коренями многочлена $f(x)$.
\end{definition}

\begin{claim}
	Многочлени $f(x)$ та $g(x)$ є взаємно простими тоді і тільки тоді, коли
	вони не мають жодного спільного кореня.
\end{claim}

\begin{claim}
	Якщо $x_0$ --- корінь кратності $k$ для $f(x)$, то $x_0$ є коренем кратності $k-1$
	для $f'(x)$.
\end{claim}
\begin{proof}
	Якщо $x_0$ --- корінь кратності $k$ для $f(x)$, то $f(x) = (x-x_0)^k\varphi(x), \varphi(x) \neq 0$.
	У цьому випадку $f'(x) = k(x-x_0)^{k-1}\varphi(x) + (x-x_0)^k\varphi'(x) = (x-x_0)^{k-1}(k\varphi(x)
	+ (x-x_0)\varphi'(x)) = (x-x_0)^{k-1}\psi(x)$, де $\psi(x) = k\varphi(x) + (x-x_0)\varphi'(x)$,
	причому $\psi(x_0) = k\varphi(x_0) + 0 \varphi'(x_0) = k\varphi(x_0) \neq 0$.
\end{proof}

\begin{theorem}[Теорема Безу]
	Остача від ділення многочлена $f(x)$ на $x-b$ дорівнює $f(b)$.
\end{theorem}
\begin{proof}
	Многочлен $f(x)$ можна подати у вигляді $f(x) =  (x-b)p(x) + r$,
	де $r = const$. Тому $f(b) = r$.
\end{proof}

\begin{corollary}
	Якщо $x_0$ --- корінь многочлена $f(x)$, то остача від ділення $f(x)$ на
	$(x - x_0)$ дорівнює нулю.
\end{corollary}
\begin{proof}
	Доведення. Очевидно, що $r = f(x_0) = 0$. 
\end{proof}

\section{Основна теорема алгебри та її наслідки}

\begin{theorem}
	Будь-який многочлен $f(x)$ ненульової степені з комплексними чи
	дійсними коефіцієнтами має принаймні один дійсний чи комплексний корінь.
\end{theorem}

\begin{corollary}
	Будь-який многочлен $f(x)$ $n$-ої степені $(n \geqslant 1)$ має рівно $n$ коренів.
\end{corollary}
\begin{proof}
	Нехай $\deg f(x) = n$, тоді за основною теоремою алгебри $x_1$ --- це
	корінь многочлена $f(x)$. Маємо $f(x_1) = 0$, тобто $f(x) = (x - x_1)g_1(x)$, причому
	$\deg g_1(x) = n-1$. Якщо $n - 1 \geqslant 1$, то многочлен $g_1(x)$ має хоча б один корінь $x_2$:
	$g_1(x_2) = 0$ і $g_1(x) = (x - x_2)g_2(x)$. Повторивши аналогічні міркування потрібну
	кількість разів, отримаємо розклад многочлена $f(x)$ на лінійні множники:
	$f(x) = a_0(x-x_1)(x-x_2)...(x-x_n)$,
	а це і означає, що даний многочлен має рівно n коренів.
\end{proof}

\begin{corollary}
	Кожний многочлен $f(x)$ можна представити у вигляді:
	$f(x) = a_0(x-x_1)^{\alpha_1}(x-x_2)^{\alpha_2}...(x-x_s)^{\alpha_s}$,
	де $x_i$ --- корінь кратності $\alpha_i$, причому $\alpha_1 + \alpha_2 + ... + \alpha_s = n = \deg f(x)$ і цей розклад
	є єдиним (з точністю до порядку множників).
\end{corollary}

\begin{corollary}
	Якщо $f(x)$ і $g(x)$ --- многочлени степені $n$ і їх значення
	співпадають в $(n+1)$ різних точках, то $f(x) \equiv g(x)$.
\end{corollary}
\begin{proof}
	Нехай існують такі значення $x_1 < ... < x_{n+1}$, що $f(x_k) = g(x_k)$,
	$k = 1, ..., n+1$. Розглянемо многочлен $F(x) = f(x) - g(x)$. Очевидно, що
	$\deg F(x) \leqslant n$ і $F(x_1) = ... = F(x_{n+1} = 0)$, тобто $x_k$, $k = 1, ..., n+1$, є коренями
	многочлена $F(x)$. Таким чином, многочлен степені $\leqslant n$ має $n+1$ корінь, а це
	суперечить наслідку 1 основної теореми алгебри. Отже, $F(x) \equiv 0$, а це означає, що
	$f(x) \equiv g(x)$.
\end{proof}

\begin{corollary}
	Якщо число $z_0$ --- це комплексний корінь многочлена $f(x)$ з
	дійсними коефіцієнтами, то $\overline{z_0}$ є також коренем цього многочлена, причому їх
	кратності співпадають. 
\end{corollary}
\begin{proof}
	Нехай $z_0$ --- це комплексний корінь многочлена з дійсними
	коефіцієнтами $f(x) = a_0x^n + a_1x^{n-1} + ... + a_{n-1}x + a_n$. Тоді $f(z_0) = 0$. Оскільки
	$a_i \in \mathbb{R}$, то $a_i = \overline{a_i}$, $i = 0, 1, ..., n$. Доведемо,
	що $f(\overline{z_0}) = 0$, використовуючи
	властивості комплексно спряжених чисел:
	
	$$f(\overline{z_0}) = a_0(\overline{z_0})^n + ... + a_{n-1}\overline{z_0} + a_n = a_0\overline{z_0^n} + ... + a_{n-1}\overline{z_0} + a_n =$$
	
	$$= \overline{a_0}\overline{z_0^n} + ... + \overline{a_{n-1}}\overline{z_0} + \overline{a_n} = \overline{a_0 z_0^n} + ... + \overline{a_{n-1}z_0} + \overline{a_n} = \overline{a_0z_0^n + ... + a_{n-1}z_0 + a_n} =$$
	
	$$= \overline{f(z_0)} = \overline{0} = 0.$$
\end{proof}

\begin{corollary}
	Довільний многочлен з дійсними коефіцієнтами можна
	представити у вигляді добутку многочленів з дійсними коефіцієнтами степені, не
	вище другої.
\end{corollary}
\begin{proof}
	Довільний многочлен $f(x)$ може мати як дійсні, так і комплексні
	корені. Нехай $x_1, ..., x_s$ --- його дійсні корені кратності $\alpha_1, ..., \alpha_s$ відповідно,
	а $z_1, ..., z_t$ --- комплексні корені кратності $\beta_1, ..., \beta_t$. Згідно з наслідком 4 комплексно спряжені
	числа $\overline{z_1}, ..., \overline{z_t}$ теж будуть коренями многочлена $f(x)$ відповідної кратності. Отже,
	даний многочлен буде ділитись на $(x - x_i)^{\alpha_i}$, $i = 1, ..., s$, без остачі. Аналогічно
	$f(x)$ буде ділитись без остачі на $(x - z_j)^{\beta_j} (x - \overline{z_j})^{\beta_j}$, 
	$j = 1, ..., t$. Отже, маємо:
	
	$$(x - z_j)^{\beta_j} (x - \overline{z_j})^{\beta_j} = ((x - z_j)(x - \overline{z_j}))^{\beta_j} = (x^2 - (z_j + \overline{z_j})x + z_j\overline{z_j})^{\beta_j} =$$
	
	$$x^2 - 2(\text{Re}z)x + |z_j|^2 = x^2 - 2a_jx + b_j$$	
		
	де $a_j, b_j \in \mathbb{R}$. Тому
	
	$$f(x) = a_0(x-x_1)^{\alpha_1} ... (x-x_s)^{\alpha_s} (x^2 + a_1x + b_1)^{\beta_1} ... (x^2 + a_tx + b_t)^{\beta_t}$$	

	причому $\alpha_1 + ... + \alpha_s + 2(\beta_1 + ... + \beta_t) = n$. 
\end{proof}


