\chapter{Лінійні оператори}


\section{Лінійний простір лінійних операторів}

Нехай $L$ та $M$ --- лінійні простори над полем $K$. Розглянемо множину лінійних
операторів, які діють з $L$ у $M$. Введемо дві лінійні операції на цій множині.


Озн. Оператор $A$ дорівнює оператору $B$ ($A = B$), якщо для довільного $x \in L$
має місце рівність $A x = B x$.


Озн. Оператор $C$ називається сумою операторів $A$ і $B$ ($C = A + B$), якщо для
довільного $x \in L$ виконується $C x = (A + B)x = A x + B x$.


Тв. 1. Якщо $A$ і $B$ --- лінійні оператори, то оператор $C = A + B$ також є
лінійним.

Доведення. Для будь-яких $x, x' \in L$ виконується співвідношення:

$$C(\alpha x + \beta x') = A(\alpha x + \beta x') + b(\alpha x + \beta x') = \alpha A x + \beta A x' + \alpha B x + \beta B x' =$$

$$= \alpha(A x + B x) + \beta(A x' + B x') = \alpha C x + \beta C x'.$$


Тв. 2. Операція додавання операторів є комутативною та асоціативною, тобто
$A + B = B + A$ та $(A + B) + C = A + (B + C).$

Доведення комутативності. Для довільного $x \in L$ маємо:

$$(A + B)x = A x + B x = B x + A x = (B + A) x.$$


Асоціативність додавання доводиться аналогічно.

Оператори відносно операції додавання утворюють абелеву групу. Нулем
групи (нейтральним елементом) є нульовий оператор: $Ox = \overline{0}$.


Якщо у просторі $L$ зафіксувати базис $e = \{e_1, e_2, ..., e_n\}$, а у просторі $M$ ---
базис $f = \{f_1, f_2, ..., f_n\}$ і побудувати матриці $A_{ef}$ та $B_{ef}$ операторів $A$ та $B$ у
цих базисах, то матриця оператора $C = A + B$ обчислюється за формулою
$C_{ef} = A_{ef} + B_{ef}$.


Озн. Оператор $C$ називається добутком оператора $A$ на константу $\lambda \in K$
($C = \lambda A$), якщо для будь-якого $x \in L$ виконується співввідношення $C x = \lambda A x$.


Тв. 3. Якщо $A$ --- лінійний оператор, то оператор $C = \lambda A$ також є лінійним.

Доведення. Для довільних $x, x' \in L$ виконується співвідношення:

$$C(\alpha x + \beta x') = \lambda A(\alpha x + \beta x') = \lambda \alpha A x + \lambda \beta A x'
= \alpha (\lambda A x) + \beta(\lambda A x') = \alpha C x + \beta C x'.$$

Матриця оператора $C = \lambda A$ обчислюється за формулою $C_{ef} = \lambda A_{ef}$.

Неважко переконатися, що для операцій додавання та множення на константу
виконуються всі властивості, які визначають лінійний простір.


Тв. 4. Множина всіх лінійних операторів, які діють з $L$ у $M$ є лінійним
простором.


\textit{Множення операторів}


Нехай оператори $A$ і $B$ діють таким чином: $B: L \rightarrow M$, $A: M \rightarrow N$, де
$L$, $M$, $N$ --- це деякі лінійні простори.


Озн. Добутком операторів $A$ і $B$ називається оператор $C$ ($C = A B$), якщо
для довільного $x \in L$ виконується $C x = (AB)x = A(B x)$.


Оператор $C = A B$ діє з простору $L$ у простір $N$. Якщо у просторі $L$ та $M$
зафіксовано раніше вибрані базиси, а в просторі $N$ – базис $g = \{g_1, g_2, ..., g_k\}$ і
побудовано матриці $A_{fg}$, $B_{ef}$, то $C_{eg} = A_{fg} B_{ef}$. 

\section{Невироджений оператор}

Розглянемо оператор $A: L \rightarrow L$. 


Озн. Оператор $A$ називається невиродженим, якщо його ядро містить лише
нульовий елемент, тобто $\Ker A = \{\overline{0}\}$. В іншому випадку оператор $A$ називається
виродженим.


Приклади.

1. Тотожній оператор $I x = x$ є невиродженим.

2. Розглянемо оператор $P: E^2 \rightarrow E^2$, який проектує довільний вектор з $E^2$ на
вісь абсцис. Цей оператор є виродженим.


Озн. Рангом лінійного оператора називається розмірність його образу, тобто
$\rang A = \dim \im A$.


Тв. 1. Якщо оператор $A$ є невиродженим, то $\rang A = \dim L$.

Доведення. Твердження випливає з теореми про розмірності ядра і образу
лінійного оператора (див. властивості лінійних операторів):

$$\dim L = \dim \Ker A + \dim \im A$$

(враховується, що $\dim \Ker A = 0$).



Тв. 2. Якщо оператори $A$ і $B$ є невиродженими, то оператор $C = A B$ є також
невиродженим.

Доведення. Нехай $x \in \Ker C$. Тоді $C x = (A B)x = \overline{0}$. Звідси випливає, що
$(AB)x = A(Bx) = \overline{0}$. З означення невиродженості оператора $A$ випливає, що
$B x = 0$, а з невиродженості оператора $B$ слідує, що $x = \overline{0}$. Це і означає, що
$\Ker C = \{\overline{0}\}$, тобто оператор $C = A B$ є невиродженим.


Тв. 3. Якщо оператор $A$ є невиродженим і $A x = y$, то для елементу $y$ існує
єдиний прообраз $x$.

Доведення проведем від супротивного. Припустимо, що $A x = y$, $A x' = y$ і
$x \neq x'$. Тоді $Ax - Ax' = A(x - x') = 0$, тобто $x - x' \in \Ker A$. Тому $x - x' = \overline{0}$ і
$x = x'$, що суперечить припущенню.

Дане твердження дає змогу ввести поняття оберненого оператора. 

Озн. Нехай $A x = y$ і оператор $A$ є невиродженим. Оператор $A^{-1}$, що
задовольняє співвідношення $A^{-1} y = x$ називається оберненим до $A$.

Тв. 4. $A^{-1}A = A A^{-1} = I$.

Доведення. Це твердження випливає із співвідношень:

$$A^{-1}(Ax) = A^{-1}y = x \text{ або } (A^{-1}A)x = Ix.$$

Отже, $A^{-1}A = I$, де $I$ --- це тотожній оператор ($Ix = x$). Аналогічно доводиться, що
$AA^{-1} = I$.


Тв 5. Якщо невироджений оператор $A$ є лінійним, то і оператор $A^{-1}$ --- лінійний.


Доведення. Позначимо $z = A^{-1}(\alpha y + \beta y') - \alpha A^{-1} y - \beta A^{-1} y'$. Подіємо на
обидві частини цієї рівності оператором $A$:

$$Az = AA^{-1}(\alpha y + \beta y') - \alpha A A^{-1} y - \beta A A^{-1} y' = \alpha y + \beta y' - \alpha y - \beta y' = \overline{0}.$$

Звідси випливає, що $z \in \Ker A$. Але $A$ --- невироджений оператор, тому $z = \overline{0}$.

Тобто $A^{-1}(\alpha y + \beta y') = \alpha A^{-1} y + \beta A^{-1} y'$ --- це і означає лінійність оператора $A^{-1}$.


Тв 6. Оператор $A^{-1}$ є невиродженим.

Доведення. Для довільного $y \in \Ker A^{-1}$ маємо $A^{-1} y = \overline{0}$. На обидві частини
цієї рівності подіємо оператором $A$. Тоді $A A^{-1} y = A \overline{0} = \overline{0}$. Оскільки
$A A^{-1} = I$, то $y = \overline{0}$, що і доводить невиродженість оператора $A^{-1}$.


Якщо лінійному оператору $A$ відповідає в деякому базисі матриця $A'$, то
оберненому лінійному оператору $A^{-1}$ відповідатиме матриця $A'^{-1}$.


Умова невиродженості оператора $A$ еквівалентна умові існування $A'^{-1}$.

Дійсно, для знаходження ядра $A$ треба розв’язати систему $A'\overline{x} = \overline{0}$. Однорідна
система лінійних рівнянь завжди має нульовий розв’язок. Цей розв’язок буде
єдиним тоді і тільки тоді, коли $\rang A' = n$ ($n$ --- кількість невідомих, $n = \dim L$).

Оскільки матриця $A'$ має розмірність $n \times n$, то $\rang A' = n \Leftrightarrow \det A' \neq 0 \Leftrightarrow \exists A'^{-1}$.


