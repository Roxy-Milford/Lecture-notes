\chapter{Лінійний простір}  %p 59


\section{Лінійний простір}

Розглянемо різні об’єкти: вектори, многочлени, комплексні числа. Над ними
були визначені лінійні операції: множення на константу і додавання. Залежно від 
природи цих об’єктів операції задавались по-різному, але вони задовольняли одним
і тим самим законам (асоціативність, комутативність та дистрибутивність).

Якщо залишити осторонь природу конкретних об’єктів і ввести аксіоматично
визначені дві операції, які мають певні властивості, то можна побудувати загальну
теорію, що узагальнює конкретні випадки. Таким чином і виникло поняття
лінійного простору як безпосереднє узагальнення дво- та тривимірних векторних
просторів.

Нехай $K$ --- це множина дійсних чисел (або комплексних чисел), тобто $k = \mathbb{R}$
або $k = \mathbb{C}, L \neq \varnothing$. 

\begin{definition}
	\textbf{Лінійний простір $L$ над полем $K$} --- це множина деяких
	елементів, на якій введено операції додавання і множення на скаляр, які не
	виводять за межі $L : x, y \in L, \alpha \ in K \Rightarrow x + y \in L, \alpha x \in L$. Для будь-яких
	елементів $x, y, z \in L$ і будь-яких чисел $\alpha, \beta \in K$ виконуються такі аксіоми:
	\begin{enumerate}
		\item $x + y = y + x$ --- комутативність додавання;
		\item $(x+y)+z = x+(y+z)$ --- асоціативність додавання;
		\item $(\alpha \beta)x = \alpha (\beta x)$ --- асоціативність множення на скаляр;
		\item $(\alpha + \beta)x = \alpha x + \beta x$ --- дистрибутивність відносно скалярів;
		\item $\alpha(x + y) = \alpha x + \alpha y)$ --- дистрибутивність відносно векторів;
		\item існує нульовий елемент $\overline{0} \in L: x + \overline{0} = x$;
		\item $\forall x \in L \: \exists x'$ --- протилежний елемент, такий що $x + x' = \overline{0}$;
		\item $1 x = x$.
	\end{enumerate}
\end{definition}

Елемент лінійного простору --- це \textbf{вектор}, сам простір ---
\textbf{векторний простір}. Лінійний простір --- це \textbf{дійсний простір}, якщо $k = \mathbb{R}$,
а якщо $k = \mathbb{C}$, то це --- \textbf{комплексний простір}.

\begin{remark}
	Символом $\overline{0}$ будемо позначати нульовий вектор лінійного
	простору, а символом 0 --- дійсне число --- нуль.
\end{remark}

Прикладами лінійного простору може слугувати множина дійсних чисел $\mathbb{R}$,
множина комплексних чисел $\mathbb{C}$, множина векторів на площині $E^2$, множина 
векторів у просторі $E^3$. Множина $C_{[a,b]}$ всіх неперервних функцій $f(x)$, які
визначені на відрізку $[a,b]$, з поточковим додаванням і множенням на константу, а
також множина многочленів $P_n$ степені $\leqslant n$ теж є лінійними просторами.

\textit{Властивості лінійного простору}
\begin{enumerate}
	\item В довільному лінійному просторі існує єдиний нульовий елемент.
	\item В довільному лінійному просторі для будь-якого елемента існує єдиний обернений.
	\item В довільному лінійному просторі $0 x = \overline{0}$, а протилежний елемент $x' = (-1)x$.
\end{enumerate}
\begin{proof}
	$$0x = 0 x + \overline{0} = 0 x + (x+x') = (0x + x) + x' = (0x + 1x) + x' = $$
	
	$$= (0 + 1)x + x' = 1x + x' = x + x' = \overline{0}.$$
\end{proof}

\begin{example}
	Проведемо узагальнення поняття вектора. Нехай $x = \begin{pmatrix}
		x_1  \\
		\vdots  \\
		x_n  \\
	\end{pmatrix}$ --- це
	впорядкована множина з $n$ дійсних чисел, записаних у стовпчик (вектор).
	Множину векторів такого виду позначимо $\mathbb{R}^n = \left\{\left.\begin{pmatrix}
		x_1  \\
		\vdots  \\
		x_n  \\
	\end{pmatrix}\right| x_i \in \mathbb{R}\right\}$. Доведемо, що $\mathbb{R}^n$
	є лінійним простором.
\end{example}
\begin{proof}
	Нехай $y \in \mathbb{R}^n$, тобто $y = \begin{pmatrix}
		y_1  \\
		\vdots  \\
		y_n  \\
	\end{pmatrix}, y_i \in \mathbb{R}m i = \overline{1, n}$, i $\alpha \in \mathbb{R}$. Тоді
	природньо ввести операції додавання та множення на скаляр таким чином:
		
	 $$x + y = \begin{pmatrix}
		x_1 + y_1  \\
		\vdots  \\
		x_n + y_n  \\
	\end{pmatrix}, \alpha x = \begin{pmatrix}
		\alpha x_1  \\
		\vdots  \\
		\alpha x_n  \\
	\end{pmatrix}.$$
	
	Окрім того, задамо вектор, протилежний до $x$, та нульовий вектор:
	
	$$x' =\begin{pmatrix}
		-x_1  \\
		\vdots  \\
		-x_n  \\
	\end{pmatrix}, \overline{0} = \begin{pmatrix}
		0  \\
		\vdots  \\
		0  \\
	\end{pmatrix}.$$

	Очевидно, що для заданих таким чином векторів і операцій додавання та
	множення на константу, виконуються всі аксіоми лінійного простору. Отже, $\mathbb{R}^n$ ---
	це лінійний простір, який називають арифметичним. 
\end{proof}

\section{Матриці як приклад лінійного простору}

\begin{definition}
	Матрицею розмірів $n$ на $m$ називається сукупність $n \times m$ чисел,
	записаних у вигляді прямокутної таблиці, що містить $n$ рядків та $m$ стовпчиків.
\end{definition}

Вектор-стовпчик можна розглядати, як матрицю $n \times 1$, матриця $1 \times n$ --- це
вектор-рядок.

Розглянемо множину матриць $M_{n \times m}, A, B, C \in M_{n \times m}$,

$$A = (a_{ij}), B = (b_{ij}), C = (c_{ij}), i = \overline{1,n},j = \overline{1,m}.$$

\begin{definition}
	$A = B \Leftrightarrow a_{ij} = b_{ij}, \forall i, j$.
\end{definition}

\begin{definition}
	\textbf{Сума матриць $A$ і $B$} --- це матриця $C = A + B$, елементи
	якої дорівнюють сумі відповідних елементів матриць $A$ і $B$, тобто $c_{ij} = a_{ij} + b_{ij}$.
\end{definition}

\begin{definition}
	\textbf{Добуток матриці $A$ на число $\alpha$} --- це матриця $B = \alpha A$, кожен
	елемент якої є добутком відповідного елемента даної матриці на число $\alpha$, тобто
	$b_{ij} = \alpha a_{ij}$.
\end{definition}

\begin{definition}
	Матриця називається нульовою, якщо всі її елементи дорівнюють нулю:

	$$O = \begin{pmatrix}
		0 & ... & 0  \\
		\vdots & \ddots & \vdots  \\
		0 & ... & 0  \\
	\end{pmatrix}.$$
\end{definition}

\begin{definition}
	Матриця $(-1)A = -A$ --- це \textbf{протилежна матриця} до матриці $A$.
\end{definition}

Очевидно, $A + (-A) = O$.

\begin{definition}
	Сума матриць $A$ і $-B$ --- це \textbf{різниця матриць $A$ і $B$} та
	позначається $A - B$. 
\end{definition}

Легко перевірити, що всі аксіоми лінійного простору виконуються, тому
множина матриць із заданими операціями додавання і множення на число є
лінійним простором.

\textit{Множення матриць}

Добуток матриць $A B$ визначається тільки за умови, коли кількість
стовпчиків матриці $A$ дорівнює кількості рядків матриці $B$.

\begin{definition}
	Добуток $A B$ матриць $A = (a_{ij})_{i=\overline{1,n}, j=\overline{1,m}}$ i $B = (b_{jk})_{j=\overline{1,m}, k=\overline{1,p}}$
	називається матриця $C = (c_{ik})_{i=\overline{1,n}, k=\overline{1,p}}$, елементи якої визначаються рівністю:
	$c_{ik} = a_{i1}b_{1k} + a_{i2}b_{2k} + ... + a_{im}b_{mk}, i = \overline{1,n}, k = \overline{1,p}$.
\end{definition}

Таким чином, елемент $c_{ik}$ дорівнює сумі добутків елементів $i$-го рядка
матриці $A$ на відповідні елементи $k$-го стовпчика матриці $B$.

\begin{example}
Приклад. Знайти добуток матриць
$A = \begin{pmatrix}
	2 & -1 \\
	0 & 3 \\
	1 & 0 \\
\end{pmatrix}$ та $B = \begin{pmatrix}
	3 & 1 \\
	0 & -1 \\
\end{pmatrix}$

$C = \begin{pmatrix}
	2 & -1 \\
	0 & 3 \\
	1 & 0 \\
\end{pmatrix} \times \begin{pmatrix}
	3 & 1 \\
	0 & -1 \\
\end{pmatrix} = \begin{pmatrix}
	2 \cdot 3 + (-1) \cdot 0 	& 2 \cdot 1 + (-1) \cdot (-1) \\
	0 \cdot 3 					& 0 \cdot 1 + 3 \cdot (-1) \\
	1 \cdot 3 + 0 \cdot 0 		& 1 \cdot 1 + 0 \cdot (-1) \\
\end{pmatrix} = \begin{pmatrix}
	6 & 3 \\
	0 & -3 \\
	3 & 1 \\
\end{pmatrix}$
\end{example}

Зауважимо, що для даних матриць знайти добуток $B \cdot A$ неможливо.

Якщо ж матриці $A$ і $B$ є квадратними розміру $n \times n$, то можна обчислити $A \cdot B$
та $B \cdot A$. Відзначимо важливу властивість: множення матриць є некомутативним,
тобто $A \cdot B \neq B \cdot A$. Проте існують матриці, які називаються переставними, для
яких виконується рівність $A \cdot B = B \cdot A$.

\begin{definition}
	Квадратна матриця, у якої всі елементи $a_{ij} = 0$ при $i \neq j$, називається
	діагональною. Якщо ж у діагональній матриці, всі елементи $a_{ii}$ дорівнюють
	одиниці, то матриця називається одиничною і позначається літерою $I$: 
	
	$$I = \begin{pmatrix}
		1 & ... & 0  \\
		 & \ddots &   \\
		 0 & ... & 1  \\
	\end{pmatrix}.$$
\end{definition}

Можна довести, що одинична матриця $I$ є переставною з будь-якою
квадратною матрицею $A$ такого ж розміру: $A I = I A = A$.

\textit{Властивості множення матриць}
\begin{enumerate}
	\item $AB \neq BA$.
	\item $(AB)C = A(BC)$.
	\item $(A+B)C = AC + BC, A(B+C) = AB + AC$.
	\item $\alpha(AB) = (\alpha A)B$.
\end{enumerate}

\textit{Лінійний підпростір}

\begin{definition}
	Підмножина $L_1$ лінійного простору $L$ називається лінійним
	підпростором, якщо вона є лінійним простором відносно операцій, введених в $L$.
\end{definition}

\begin{claim}
	$L_1$ є лінійним підпростором $L \Leftrightarrow$
	
	$\Leftrightarrow \begin{array}{l}
		1) \forall x, y \in L_1 : x + y \in L_1  \\
		2) \forall x \in L_1, \alpha \in \mathbb{R} : \alpha x \in L_1  \\
	\end{array}$
\end{claim}

\begin{proof}
	Якщо $L_1$ --- лінійний підпростір $L$ , то умови $1)$ та $2)$ виконуються
	автоматично. Покажемо, що підмножина $L_1$, елементи якої задовольняють умови
	$1)$ та $2)$ є лінійним підпростором. Зауважимо, що всі аксіоми, окрім 6 та 7,
	виконуються, оскільки вони відносяться до будь-яких елементів простору L.
	
	Перевіримо, чи виконуються аксіоми 6 та 7.
	
	Нехай $x \in L_1$. Згідно з умовою $2)$ маємо $\alpha x \in L_1$. Оскільки $\alpha \in \mathbb{R}$ є довільним,
	то при $\alpha = 0$ та $\alpha = -1$ маємо співвідношення: $0x = \overline{0} \in L_1$, $(-1)x = -x \in L_1$. Це
	означає, що в $L_1$ існує нульовий елемент $\overline{0}$ і для довільного $x \in L_1$ існує
	протилежний елемент $x' = -x \in L_1$. Тому аксіоми 6 та 7 також виконуються. Отже,
	$L_1$ є лінійним простором, тобто $L_1$ --- це лінійний підпростір лінійного простору $L$.
\end{proof}

\textit{Приклади лінійних підпросторів:}
\begin{enumerate}
	\item множина ${\overline{0}}$ --- це лінійний підпростір для будь-якого лінійного простору $L$;
	\item множина компланарних векторів $E^2$ є лінійним підпростором простору $E^3$;
	\item множина колінеарних векторів $E^1$ є лінійним підпростором простору $E^2$ і простору $E^3$. 
\end{enumerate}

\section{Лінійні оболонки}

Нехай $L$ --- це довільний лінійний простір, $a_1, a_2, ..,a_n \in L$.

\begin{definition}
	Лінійною оболонкою векторів $a_1, a_2, ..,a_n$ називається множина всіх
	можливих лінійних комбінацій цих векторів:
	
	$$\Lambda(a_1, a_2, ..,a_n) = \{x \mid x \in L, x = \sum\limits_{i=1}^n \alpha_i a_i, \alpha_i \in K\}$$
\end{definition}

\begin{example}
	\begin{enumerate}
		\item $\overline{a} \in E^3, \overline{a} \neq \overline{0}, \Lambda(\overline{a}) = \{x \mid x \in E^3, x = \alpha \overline{a}\} = E^1$.
		\item Вектори $\overline{a}, \overline{b}$ --- неколінеарні, $\Lambda(\overline{a}, \overline{b}) = \{x \mid x = \alpha \overline{a} + \beta \overline{b}, x \in E^3\} = E^2$.
	\end{enumerate}
\end{example}


\begin{claim}
	Лінійна оболонка довільних векторів $a_1, a_2, ..,a_n$ з $L$ є лінійним
	підпростором цього ж лінійного простору $L : \Lambda(a_1, a_2, ..,a_n) \subset L$.
\end{claim}
\begin{proof}
	Нехай $x, y \in \Lambda(a_1, a_2, ..,a_n)$, тоді $x = \sum\limits_{i=1}^n \alpha_i a_i$,
	$y = \sum\limits_{i=1}^n \beta_i a_i$.
	
	Скористаємось твердженням про лінійний підпростір, доведемо виконання
	двох умов.
	
	$$1) \lambda x = \lambda \sum\limits_{i=1}^n \alpha_i a_i = \sum\limits_{i=1}^n (\lambda \alpha_i) a_i \in \Lambda(a_1, ..., a_n).$$
	
	$$2) x+y = \sum\limits_{i=1}^n \alpha_i a_i + \sum\limits_{i=1}^n \beta a_i = \sum\limits_{i=1}^n (\alpha_i + \beta_i) a_i \in \Lambda(a_1, ..., a_n).$$
\end{proof}

\section{Базис і розмірність лінійного простору}

Нехай $L$ --- це довільний лінійний простір, $L \neq \{0\}$. 

\begin{definition}
	Базисом лінійного простору $L$ називається максимальна лінійно
	незалежна система векторів (приєднання будь-якого вектора з $L$ перетворює цю
	систему на лінійно залежну). Якщо базис містить $n$ векторів, то лінійний простір
	$L$ називається $n$-вимірним, а число $n$ --- розмірністю лінійного простору і
	позначається $\dim L = n$.
\end{definition}

\begin{remark}
	Якщо $L$ містить лише $\overline{0}$, то $\dim L = 0$.
\end{remark}

\textit{Алгоритм побудови базису формулюється таким чином.}
\begin{enumerate}
	\item Вибираємо довільний вектор $e_1 \in L, e_1 \neq \overline{0}$.
	
	\item Вибираємо довільний вектор $e_2 \in L, e_2 \neq \overline{0}$ таким чином, щоб вектори $e_1$ та
	$e_2$ були лінійно незалежними.
	
	\item Припустимо, що вказана процедура дозволила побудувати лінійно незалежні
	вектори $e_1, e_2, ..., e_n$. Якщо приєднання до $e_1, e_2, ..., e_n$ довільного вектора $b \in L$
	перетворює цю систему векторів на лінійно залежну, то вектори $e_1, e_2, ..., e_n$
	утворюють базис лінійного простору $L$. Тоді $b = \sum\limits_{i=1}^n \alpha_i e_i$ і $b \in \Lambda(e_1, e_2, ..., e_n)$.
\end{enumerate}

\begin{claim}
	Якщо вектори $e_1, e_2, ..., e_n$ є базисом лінійного простору $L$, то
	$\Lambda(e_1, e_2, ..., e_n) = L$.
\end{claim}

\begin{definition}
	Система векторів простору $L$ називається повною, якщо кожний вектор
	простору $L$ лінійно виражається через вектори цієї системи.
\end{definition}

\begin{definition}
	Базисом лінійного простору $L$ називається повна лінійно незалежна
	система векторів.
\end{definition}

Доведемо еквівалентність двох означень базису лінійного простору.
\begin{proof}
	Озн. 1 $\rightarrow$ Озн. 2. Нехай $e_1, e_2, ..., e_n$ --- максимальна лінійно незалежна система
	векторів. Тоді для будь-якого $x \in L$ вектори $e_1, e_2, ..., e_n, x$ є лінійно залежними, і
	можна довести, що $x = \sum\limits_{i=1}^n x_ie_i$. Отже, $e_1, e_2, ..., e_n$ --- це повна
	лінійно незалежна система векторів.

	Озн. 2 $\rightarrow$ Озн. 1. Нехай $e_1, e_2, ..., e_n$ --- це повна лінійно незалежна система
	векторів. Тоді будь-який вектор $x \in L$ можна подати у вигляді $x = \sum\limits_{i=1}^n x_ie_i$, тобто
	$\sum\limits_{i=1}^n x_ie_i + (-1)x = \overline{0}$. Отже, вектори $e_1, e_2, ..., e_n, x$ є лінійно
	залежними. Це означає, що $e_1, e_2, ..., e_n$ --- це максимальна лінійно незалежна система.
\end{proof}

\begin{claim}
	Розмірність лінійного простору не залежить від вибору базису, тобто
	кількість векторів в різних базисах – однакова.
\end{claim}


\begin{example}
Базиси в лінійних просторах просторах $E^1, E^2, E^3$ відомі.

1. Канонічним базисом лінійного простору $\mathbb{R}^n = \left\{\left.\begin{pmatrix}
		x_1  \\
		\vdots  \\
		x_n  \\
	\end{pmatrix}\right| x_i \in \mathbb{R}\right\}$ є система
	векторів $e_1 = \begin{pmatrix}
		1  \\
		0  \\
		\vdots  \\
		0  \\
	\end{pmatrix}$, $e_2 = \begin{pmatrix}
		0  \\
		1  \\
		\vdots  \\
		0  \\
	\end{pmatrix}$, ...,$e_n = \begin{pmatrix}
		0  \\
		0  \\
		\vdots  \\
		1  \\
	\end{pmatrix}$. Доведемо, що ця система векторів є
лінійно незалежною і повною, тобто може слугувати базисом $\mathbb{R}^n$.

a) Доведемо лінійну незалежність векторів $e_1, ..., e_n$ від супротивного.
Припустимо, що вектори $e_1, ..., e_n$ --- лінійно залежні. Тоді
$\exists \alpha_i, i = \overline{1,n}$ $\sum\limits_{i=1}^n |\alpha_i| \neq 0$ такі, що $\alpha_1 e_1 + \alpha_2 e_2 + ... + \alpha_n e_n = \overline{0}$. Отже,

$\alpha_1 \begin{pmatrix}
		1  \\
		0  \\
		\vdots  \\
		0  \\
	\end{pmatrix} + \alpha_2 \begin{pmatrix}
		0  \\
		1  \\
		\vdots  \\
		0  \\
	\end{pmatrix} + ... + \alpha_n \begin{pmatrix}
		0  \\
		0  \\
		\vdots  \\
		1  \\
	\end{pmatrix} 
	=
	\begin{pmatrix}
		\alpha_1  \\
		0  \\
		\vdots  \\
		0  \\
	\end{pmatrix} + \begin{pmatrix}
		0  \\
		\alpha_2  \\
		\vdots  \\
		0  \\
	\end{pmatrix} + ... + \begin{pmatrix}
		0  \\
		0  \\
		\vdots  \\
		\alpha_n  \\
	\end{pmatrix}
	=
	\begin{pmatrix}
		\alpha_1  \\
		\alpha_2  \\
		\vdots  \\
		\alpha_n  \\
	\end{pmatrix}
	=
	\begin{pmatrix}
		0  \\
		0  \\
		\vdots  \\
		0  \\
	\end{pmatrix}.$

	Отже, $\alpha_1 = \alpha_2 = ... = \alpha_n = 0$, а це суперечить припущенню. Отже,
	вектори $e_1, ..., e_n$ є лінійно незалежними.
	
	b) Доведемо повноту системи векторів $e_1, ..., e_n$. Виберемо довільний вектор
	$x \in \mathbb{R}^n, x = \begin{pmatrix}
		x_1  \\
		x_2  \\
		\vdots  \\
		x_n  \\
	\end{pmatrix}$. Тоді
	
	$x = \begin{pmatrix}
		x_1  \\
		x_2  \\
		\vdots  \\
		x_n  \\
	\end{pmatrix} = \begin{pmatrix}
		x_1  \\
		0  \\
		\vdots  \\
		0  \\
	\end{pmatrix} + \begin{pmatrix}
		0  \\
		x_2  \\
		\vdots  \\
		0  \\
	\end{pmatrix} + ... + \begin{pmatrix}
		0  \\
		0  \\
		\vdots  \\
		x_n  \\
	\end{pmatrix} 
	=
	x_1 \begin{pmatrix}
		1  \\
		0  \\
		\vdots  \\
		0  \\
	\end{pmatrix} + x_2 \begin{pmatrix}
		0  \\
		1  \\
		\vdots  \\
		0  \\
	\end{pmatrix} + ... + x_n \begin{pmatrix}
		0  \\
		0  \\
		\vdots  \\
		1  \\
	\end{pmatrix}
	= x_1e_1 + x_2e_2 + ... + x_ne_n.$	

	Довільний вектор $x$ лінійно виражається через вектори $e_1, ..., e_n$. Отже, дана
	система векторів є повною і $\dim \mathbb{R}^n = n$ --- це кількість векторів в базисі.

	2. $P_n = \{f(t) \mid f(t) = a_0 + a_1t + ... + a_nt^n\}$ --- це лінійний простір многочленів
	степені $\leqslant n$. Канонічний базис $P_n: e_0=1, e_1=t, ..., e_n=t^n$, $\dim P_n = n+1$.

	3. $M_{2 \times 3}$ --- лінійний простір матриць розміру $2 \times 3$. Канонічний базис простору	
	$M_{2 \times 3}:$
	$e_1 = \begin{pmatrix}
		1 & 0 & 0 \\
		0 & 0 & 0 \\
	\end{pmatrix}$, $e_2 = \begin{pmatrix}
		0 & 1 & 0 \\
		0 & 0 & 0 \\
	\end{pmatrix}$, ..., $e_6 = \begin{pmatrix}
		0 & 0 & 0 \\
		0 & 0 & 1 \\
	\end{pmatrix}$, $\dim M_{2 \times 3} = 2 \cdot 3 = 6$.
\end{example}

\section{База і ранг системи векторів}  %p 68

Нехай вектори $S = \{a_1, a_2, ..., a_n\}$ є елементами деякого лінійного простору $L$.

\begin{definition}
	Базою системи векторів $S$ називають максимальну лінійно незалежну
	підсистему $\{a_1, a_2, ..., a_k\}$, $k \leqslant n$, $a_i \in S$, $i = \overline{1,k}$.
\end{definition}

Це означає, що всі вектори $a_i$, де $i \geqslant k+1$ лінійно виражаються через $k$
векторів бази.

\begin{definition}
	Рангом системи векторів $S$ називають кількість векторів у базі та
	позначають $\rang S = k$.
\end{definition}

Доведемо наступне твердження.

\begin{claim}
	Нехай $\Lambda_1 = \Lambda(a_1, a_2, ..., a_n)$ --- лінійна оболонка векторів
	системи $S$, $\Lambda_2 = \Lambda(a_1, a_2, ..., a_k)$ --- лінійна оболонка векторів бази.
	Тоді $\Lambda_1 = \Lambda_1$.
\end{claim}
\begin{proof}
	Нехай $t_1 \in \Lambda$, тобто $t_1 = \alpha_1 a_1 + ... + \alpha_k a_k + \alpha_{k+1} a_{k+1} + ... + \alpha_n a_n$.
	Оскільки всі вектори $a_{k+1}, ..., a_n$ лінійно виражаються через вектори бази
	$a_1, a_2, ..., a_k$, то $t_1 = \gamma_1 a_1 + ... + \gamma_k a_k \in \Lambda_2$ і $\Lambda_1 \subseteq \Lambda_2$.

	Якщо ж $T_2 \in \Lambda_2$, то

	$t_2 = \beta_1 a_1 + ... + \beta_k a_k = \beta_1 a_1 + ... + \beta_k a_k + 0 a_{k+1} + ... + 0 a_{n} \in \Lambda_1$ і $\Lambda_2 \subseteq \Lambda_1$.

	Отже, $\Lambda_1 = \Lambda_2$.
\end{proof}

\begin{problem}
	Знайти базу системи векторів $x_1 = (0,2,-1)$, $x_2 = (3,7,1)$, $x_3 = (2,0,3)$, $x_4 = (5,1,8)$.
\end{problem}
\begin{solution}
	Вектори $x_1$ та $x_2$ --- лінійно незалежні, оскільки їх координати не
	пропорційні. Розглянемо трійку векторів $x_1$, $x_2$, $x_3$. Як відомо, цим елементам з
	простору $R^3$ можна поставити у взаємно однозначну відповідність вектори
	простору $E^3$ з такими ж координатами у базисі $\overline{i}$, $\overline{j}$, $\overline{k}$. З’ясуємо, чи буде вказана
	трійка векторів некомпланарною, обчисливши їх мішаний добуток:

	$$x_1 x_2 x_3 = \left|\begin{matrix}
		0 & 2 & -1 \\
		3 & 7 & 1 \\
		2 & 0 & 3 \\
	\end{matrix}\right| = \left|\begin{matrix}
		0 & 0 & -1 \\
		3 & 9 & 1 \\
		2 & 6 & 3 \\
	\end{matrix}\right| = 0.$$

	Таким чином, вектори $x_1$, $x_2$, $x_3$ є компланарними, тобто лінійно залежними і
	не можуть складати базу. Розглянемо іншу трійку векторів, наприклад, $x_1$, $x_2$, $x_4$.
	Аналогічно,
	
	$$x_1 x_2 x_4 = \left|\begin{matrix}
		0 & 2 & -1 \\
		3 & 7 & 1 \\
		5 & 1 & 8 \\
	\end{matrix}\right| = \left|\begin{matrix}
		0 & 0 & -1 \\
		3 & 9 & 1 \\
		2 & 17 & 8 \\
	\end{matrix}\right| = - \left|\begin{matrix}
		3 & 9 \\
		5 & 17 \\
	\end{matrix}\right| = -3 \left|\begin{matrix}
		1 & 3 \\
		5 & 17 \\
	\end{matrix}\right| = (-3) 2 = -6 \neq 0.$$

	Це означає, що $x_1$, $x_2$, $x_4$ --- лінійно незалежні та можуть слугувати базою.
	Зауважимо, що ранг даної системи векторів не може бути більше трьох, бо
	$\dim R^3 = \dim E^3 = 3$.
\end{solution}

\section{Лінійні оператори}

Нехай $L$, $M$ --- довільні лінійні простори.

\begin{definition}
	\textbf{Лінійний оператор $A: L \rightarrow M$} --- це відображення простору
	$L$ у простір $M$ таке, що $\forall x \in L: Ax = y$, $y \in M$, і для довільних $\alpha \in \mathbb{R}$ та $x' \in L$
	виконуються співвідношення: 
	\begin{enumerate}
		\item $A(\alpha x) = \alpha A x$;
		\item $A(x + x') = A x + A x'$.
	\end{enumerate}
\end{definition}

Лінійні оператори іноді називають лінійними перетвореннями одного
простору в інший.

\begin{definition}
	Якщо $Ax = y$, то вектор $y$ --- це \textbf{образ вектора $x$}, а вектор $x$
	--- це \textbf{прообраз $y$} при лінійному відображенні простору $L$ у простір $M$.
\end{definition}

\textit{Приклади лінійних операторів.}
\begin{example}
	1. Нульовий оператор $O: L \rightarrow L$, $Ox = \overline{0}$. Нульовий оператор $O$ переводить будь-який вектор простору $L$ у нульовий вектор.
	
	2. Тотожний оператор $I: L \rightarrow L$, $Ix = x$.

	3. Оператор подібності $A: L \rightarrow L$, $Ax = \lambda x$, $\lambda = const \neq 0$.
\end{example}

\begin{proof}
	Доведемо, що оператор подібності є лінійним. Перевіримо дві властивості лінійного оператора:
	
	а) $A(\alpha x) = \lambda(\alpha x) = \alpha(\lambda x) = \alpha(A x)$;

	б) $A(x + x') = \lambda(x + x') = \lambda x + \lambda x' = A x + A x'$.
\end{proof}

\begin{definition}
	\textbf{Ядро лінійного оператора $A$} --- це множина
	$\Ker A = \{x \mid x \in L, Ax = \overline{0}\}$.
\end{definition}

\begin{definition}
	\textbf{Образ лінійного оператора $A$} ---це множина
	$\Im A = \{y \mid \exists x \in L: Ax = y, y \Im M\}$.
\end{definition}

\begin{example}
	1. Для нульового оператора $O: \Ker O = L$, $\Im O = \{\overline{0}\}$.
	
	2. Для тотожнього оператора $I: \Ker I = \{\overline{0}\}$, $\im I = L$.

	3. Нехай $D$ --- це оператор диференціювання, $D: P_n \rightarrow P_{n-1}$, $Df(t) = f'(t)$.
\end{example}

Доведемо, що заданий оператор є лінійним, використовуючи правила
диференціювання функцій: 
\begin{proof}
	$$D(\alpha f(t)) = (\alpha f(t))' = \alpha f'(t) = \alpha D f(t),$$

	$$D(f(t) + g(t)) = (f(t) + g(t))' = f'(t) + g'(t) = D f(t) + D g(t)$$
\end{proof}


Знайдемо ядро та образ оператора $D$:

$$\Ker D = \{f(t): f(t) = const\}, \im D = P_{n-1}.$$


\textit{Властивості лінійних операторів}
\begin{enumerate}
	\item Лінійний оператор $A$ нульовий вектор простору $L$ переводить у нульовий
	вектор простору $M$.
	\begin{proof}
		Нехай $\overline{0} \in L$, тоді $A\overline{0} = A(0 x) = o A x = \overline{0}_M$, де $\overline{0}_M$ --- це нульвектор простору $M$.
	\end{proof}

	\item $A(-x) = -A x$.
	\begin{proof}
		$A(-x) = A(-1x) = -1 A x = -Ax$.
	\end{proof}

	\item Лінійний оператор переводить будь-яку лінійну комбінацію векторів з
	простору $L$ у таку ж лінійну комбінацію відповідних образів цих векторів, тобто
	$$A\left( \sum\limits_{i=1}^n \alpha_i x_i \right) = \sum\limits_{i=1}^n \alpha_i A x_i.$$
\end{enumerate}

\begin{claim}
	Ядро і образ лінійного оператора $A$ є лінійними підпросторами просторів
	$L$ та $M$ відповідно.
\end{claim}
\begin{proof}
	Доведення. У кожному з випадків перевіримо виконання двох умов (при
	доведенні другого пункту дві умови об’єднані в одну).

	Нехай $x \in \Ker A$, $x' \in \Ker A$. За означенням ядра $A x = \overline{0}$ та $A x' = \overline{0}$. Тоді
	$A(\alpha x) = \alpha A x = \alpha \overline{0} = \overline{0}$, тобто $\alpha x \in \Ker A$. Перевіримо другу умову:
	$A(x + x') = A x + A x' = \overline{0} + \overline{0} = \overline{0}$, тобто $(x + x') \in \Ker A$. Отже, $\Ker A$ є
	підпростором лінійного простору $L$.

	Нехай $y, y' \in \im A$. Тоді $\exists x, x' \in L: Ax = y, Ax' = y'$. Очевидно, що
	$\alpha x + \beta x' \in L$, тобто $A(\alpha x + \beta x') \in \im A$. Однак $A(\alpha x + \beta x') = \alpha y + \beta y'$. Це
	означає, що $\alpha y + \beta y' \in \im A$. Отже, $\im A$ є підпростором лінійного простору $M$. 
\end{proof}

\begin{definition}
	Розмірність образу оператора $A$ --- це \textbf{ранг оператора}:
	$\dim \im A = \rang A$.
\end{definition}

\begin{theorem}
	Для довільного лінійного оператора $A: L \rightarrow M$ справедлива
	формула $\dim L = \dim \Ker A + \dim \im A$.
\end{theorem}
\begin{proof}
	Нехай $\dim L =n$, $\dim \Ker A = k$, $k \leqslant n$, вектори
	$\underbrace{e_1, e_2, ..., e_k}\limits_{\text{Базис } \Ker A}$, $e_{k+1}$, $...$, $e_n$ --- базис простору $L$,
	а вектори $e_1, e_2, ..., e_k$ слугують базисом
	ядра оператора $A$. Побудуємо вектори $f_1 = A e_{k+1}$, $f_2 = A e_{k+2}$, ..., $f_{n-k} = A e_{n}$.

	Доведемо, що вектори $f_1, f_2, ..., f_{n-k}$ утворюють базис образу оператора $A$.
	Для цього доведемо їх лінійну незалежність (від супротивного) та повноту.

	Припустимо, що існують такі константи $\alpha_1$, $\alpha_2$, $...$, $\alpha_{n-k}$,
	$\sum\limits_{i=1}^{n-k}|\alpha_i| \neq 0$, що
	$\alpha_1 f_1 + \alpha_2 f_2 + ... + \alpha_{n-k} f_{n-k} = \overline{0}$. Це
	співвідношення можна переписати у вигляді:
	$\alpha_1 A e_{k+1} + \alpha_2 A e_{k+2} + ... + \alpha_{n-k} A e_{n} = A(\alpha_1 e_{k+1} + \alpha_2 e_{k+2} + ... + \alpha_{n-k} e_{n}) = \overline{0}$.

	Оскільки $\alpha_1 e_{k+1} + ... + \alpha_{n-k} e_{n} \notin \Ker A$, то $\alpha_1 e_{k+1} + ... + \alpha_{n-k} e_{n} = \overline{0}$. Вектори
	$e_{k+1}, ..., e_{n}$ лінійно незалежні, тому $\alpha_1 = \alpha_2 = ... = \alpha_{n-k} = 0$. Це і означає, що
	вектори $f_1, f_2, ..., f_{n-k}$ лінійно незалежні.

	Нехай $y \in \im A$, тобто існує таке $x \in L$, що $Ax = y$. Розкладемо вектор $x$ за
	базисом: $x = \beta_1 e_1 + ... + \beta_k e_k + \alpha_1 e_{k+1} + ... + \alpha_{n-k} e_{n}$. Тоді

	$$y = Ax = A(\beta_1 e_1 + ... + \beta_k e_k + \alpha_1 e_{k+1} + ... + \alpha_{n-k} e_{n}) =$$

	$$\beta_1 A e_1 + ... + \beta_k A e_k + \alpha_1 A e_{k+1} + ... + \alpha_{n-k} A e_{n}.$$

	Оскільки вектори $e_1, e_2, ..., e_k \in \Ker A$, то $\beta_1 A e_1 + ... + \beta_k A e_k = \overline{0}$, тобто
	$y = \alpha_1 A e_{k+1} + ... + \alpha_{n-k} A e_{n} = \alpha_1 f_1 + ... + \alpha_{n-k} f_{n-k}$,
	а це і означає, що система векторів $f_1, ..., f_{n-k}$ --- повна.

	Отже, вказані вектори можуть слугувати базисом у підпросторі $\im A$, причому
	$\dim \im A = n - k$. Маємо $dim \Ker A + \dim \im A = k + (n-k) = n = \dim L$.
\end{proof}

\section{Матриця лінійного оператора}

Нехай задано лінійний оператор $A: L \rightarrow M$, причому $\dim L = n$, $\dim M = m$.
У просторі $L$ виберемо базис $\{e_1, ..., e_n\}$, у просторі $M$ --- базис $\{f_1, ..., f_n\}$.

Якщо $x \in L$, то $x = \sum\limits_{i=1}^n x_i e_i$. Подіємо лінійним оператором $A$ на кожний
базисний вектор простору $L$ і отримаємо вектори з простору $M: A e_i = \sum\limits_{j=1}^m a_{ji} f_j$,
$i = \overline{1,n}$. Знайдемо образ елемента $x$ при даному лінійному відображенні.
Враховуючи лінійність оператора $A$, маємо:

$$A x = A\left( \sum\limits_{i=1}^n x_i e_i \right)
= \sum\limits_{i=1}^n A(x_i e_i) = \sum\limits_{i=1}^n x_i A e_i
= \sum\limits_{i=1}^n x_i \sum\limits_{j=1}^m a_{ji} f_j
= \sum\limits_{i=1}^n\sum\limits_{j=1}^m x_i a_{ji} f_j = $$

$$ = \sum\limits_{i=1}^n \left( \sum\limits_{j=1}^m x_i a_{ji} \right) f_j$$

Оскільки $Ax = y$, $y \in M$, то $A x = y = \sum\limits_{j=1}^m y_j f_j$. З єдиності розкладу вектора
$y = A x$ за базисом простору $M$ випливає, що $\sum\limits_{i=1}^n a_{ji} x_i = y_j$, $j = \overline{1,m}$.
Це означає, що лінійне відображення $A$ простору $L$ у простір $M$ при фіксованих базисах
повністю визначається матрицею

$$A = \begin{pmatrix}
	a_{11} & a_{12} & ... & a_{1n} \\
	a_{21} & a_{22} & ... & a_{2n} \\
	...    & ...    & ... & ...    \\
	a_{m1} & a_{m2} & ... & a_{mn} \\
\end{pmatrix},$$

яка називається матрицею лінійного оператора $A$ у заданих базисах.

\underline{Алгоритм побудови матриці лінійного оператора}

1. Подіяти лінійним оператором $A$ на кожний базисний вектор простору $L$ по
черзі.

2. Розкласти отримані вектори за базисом простору $M$.

3. Коефіцієнти розкладу записати відповідними стовпчиками шуканої матриці.


Якщо ввести позначення $\overline{x} = \begin{pmatrix} x_1 \\  \vdots \\  x_n \\  \end{pmatrix}$,
$\overline{y} = \begin{pmatrix} y_1 \\  \vdots \\  y_n \\  \end{pmatrix}$,
$\overline{x}, \overline{y} \in \mathbb{R}^n$, то операторне
рівняння $A x = y$ можна записати у матричному вигляді $A \overline{x} = \overline{y}$.

\underline{Частковий випадок:} якщо задано лінійний оператор $A: L \rightarrow L$, то йому
відповідає квадратна матриця.

\underline{Зауваження.} Якщо потрібно побудувати матрицю лінійного оператора в
даному базисі (а не в базисах), то це саме вказаний частковий випадок.


Задача. Побудувати матрицю оператора диференціювання
$D: P_3 \rightarrow p_2$, $D f(t) = f'(t)$, якщо задано базис $P_3: \{1, t, t^2, t^3\}$
і базис $P_2: \{t^2, t, 1\}$.
За допомогою матриці знайти образ многочлена $f(t) = 3t^3 - 2t^2 + 4t -5$.


Розв’язання. За алгоритмом побудови матриці лінійного оператора маємо:

$$D1 = 1' = 0 = 0t^2 + 0t + 0~1,$$
$$Dt = t' = 1 = 0 t^2 + 0t + 1~1,$$
$$Dt^2 = (t^2)' = 2t = 0t^2 + 2t + 0~1,$$
$$Dt^3 = (t^3)' = 3t^2 = 2t^2 + 0t + 0~1.$$

Матриця даного оператора має вигляд:
$D = \begin{pmatrix}
	0 & 0 & 0 & 3 \\
	0 & 0 & 2 & 0 \\
	0 & 1 & 0 & 0 \\
\end{pmatrix}$

Даному многочлену $f(t) = 3t^3 - 2t^2 + 4t - 5$ поставимо у відповідність
вектор $\overline{f} \in \mathbb{R}^4$, координати якого є коефіцієнтами розкладу многочлена $f(t)$ за
базисом простору $P^3$. Маємо $\overline{f} = \begin{pmatrix}
	-5 \\
	4 \\
	-2 \\
	3 \\
\end{pmatrix}$
. Тоді $D \overline{f} = \begin{pmatrix}
	0 & 0 & 0 & 3 \\
	0 & 0 & 2 & 0 \\
	0 & 1 & 0 & 0 \\
\end{pmatrix} \begin{pmatrix}
	-5 \\
	4 \\
	-2 \\
	3 \\
\end{pmatrix} = \begin{pmatrix}
	9 \\
	-4 \\
	4 \\
\end{pmatrix}$.

В результаті отримали вектор $D\overline{f} \in \mathbb{R}^3$, координати якого є коефіцієнтами
розкладу образу даного многочлена $f(t)$ за базисом простору $P_2$, тобто 
$D f = 9 t^2 - 4 t + 4$. Знайшовши похідну многочлена $f(t)$, легко перевірити, що
матрицю лінійного оператора $D$ побудовано вірно. 

\section{Функціонали}

Озн. Функціоналом називається відображення лінійного простору у множину
дійсних чи комплексних чисел, тобто $\varphi: L \rightarrow \mathbb{R}$ $(\mathbb{C})$.

Озн. Функціонал $\varphi$ називається лінійним, якщо $\forall x, x' \in L$ виконуються такі
співвідношення:

1) $\varphi(\alpha x) = \alpha \varphi(x)$, $\alpha \in \mathbb{R}$;
2) $\varphi(x + x') = \varphi(x) + \varphi(x')$.

Приклади.

1. $L = E^3$, $\overline{a} = (a_1, a_2, a_3)$ --- фіксований вектор з $E^3$, $\varphi(\overline{x}) = (\overline{x},\overline{a})$.
2. $L = \mathbb{R}^2$, $x = \begin{pmatrix} x_1 \\ x_2 \\ \end{pmatrix}$, $a = \begin{pmatrix} a_1 \\ a_2 \\ \end{pmatrix}$,
$\varphi(x) = \left|\begin{matrix} x_1 & a_1 \\ x_2 & a_2 \\ \end{matrix}\right|$.


Білінійний та n-лінійний функціонали


Озн. Білінійним функціоналом називається таке відображення
$\varphi: L \times L \rightarrow \mathbb{R} (\mathbb{C})$, що при кожному фіксованому значенні одного з
аргументів він є лінійним за другим аргументом, тобто:

1) $\varphi(\alpha x, y) = \alpha \varphi(x, y) = \varphi(x,\alpha y)$, $\alpha \in \mathbb{R}$;

2) $\forall x, x', y, y' \in L \varphi(x + x',y) = \varphi(x,y) + \varphi(x',y)$,
$\varphi(x,y + y') = \varphi(x,y) + \varphi(x,y')$.


Озн. Білінійний функціонал називається симетричним, якщо для довільних
$x, y \in L$ виконується рівність $\varphi(x,y) = \varphi(y,x)$ і кососиметричним, якщо
$\varphi(x,y) = - \varphi(y,x)$.


Тв. Якщо $\varphi$ --- білінійний кососиметричний функціонал, то
$\varphi(x,x) = - \varphi(x,x) = 0$.


Озн. Функціонал $\varphi: \underbrace{L \times ... \times L} \rightarrow \mathbb{R}(\mathbb{c})$
називається $n$-лінійним, якщо він є
лінійним за кожним із своїх аргументів:

1) $\varphi(x_1, ..., \alpha x_i, ..., x_n) = \alpha \varphi(x_1, ..., x_i, ..., x_n)$, $\alpha \in \mathbb{R}$;

2) $\varphi(x_1, ..., x_i + x'_i, ..., x_n) = \varphi(x_1, ..., x_i, ..., x_n) + \varphi(x_1, ..., x'_i, ..., x_n)$.


Озн. $n$-лінійний функціонал називається симетричним, якщо для довільних
$x_i, x_j \in L$ виконується рівність  $\varphi(x_1, ..., x_i, ..., x_j, ..., x_n) = \varphi(x_1, ..., x_j, ..., x_i, ..., x_n)$
і кососиметричним, якщо $\varphi(x_1, ..., x_i, ..., x_j, ..., x_n) = - \varphi(x_1, ..., x_j, ..., x_i, ..., x_n)$.

Тв. Якщо $\varphi$ --- $n$-лінійний кососиметричний функціонал, то $\varphi(x_1, ..., x_n) = 0$,
якщо $x_i = x_j$ для деяких $1 \leqslant i < j \leqslant n$.


Приклади.

1. $x, y \in E^3$, $\varphi(\overline{x},\overline{y}) = (\overline{x},\overline{y})$ --- скалярний
добуток двох векторів --- це симетричний білінійний функціонал.

2. $x, y \in E^2$, $\varphi(x,y) = \left| \begin{matrix} x_1 & y_1 \\ x_2 & y_2 \\ \end{matrix} \right|$ ---
це білінійний кососиметричний функціонал.

3. $x, y, z \in E^3$,
$\varphi(x,y,z) = \left| \begin{matrix}
	x_1 & y_1 & z_1 \\
	x_2 & y_2 & z_2 \\
	x_3 & y_3 & z_3 \\
\end{matrix} \right|$
--- це трилінійний кососиметричний функціонал.


\underline{Лінійні функціонали в лінійному просторі з фіксованим базисом
(частковий випадок)}


Нехай $L$ --- це тривимірний лінійний простір, у якому зафіксовано базис
$\{e_1, e_2, e_3\}$. Якщо $x_1, x_2, x_3 \in L$, то

$$x_1 = a_{11} e_1 + a_{12} e_2 + a_{13} e_3,$$

$$x_2 = a_{21} e_1 + a_{22} e_2 + a_{23} e_3,$$

$$x_1 = a_{31} e_1 + a_{32} e_2 + a_{33} e_3.$$


Якщо $\varphi(x_1, x_2, x_3)$ --- трилінійний кососиметричний функціонал, то 

$$\varphi(x_1, x_2, x_3) = $$




$$(1)$$





Озн. Функціонал називається нормованим, якщо $\varphi(e_1, e_2, e_3) = 1$.

\underline{Перестановки та підстановки}

Нехай $A = \{1, ..., n\}$.

Озн. Перестановкою $n$-порядку називається будь-який впорядкований набір
чисел множини $A$.


Тв. Кількість усіх перестановок $n$-порядку становить $n!$.

Озн. Елемент перестановки $a_i$ утворює інверсію з елементом $a_j$, $i<j$, якщо
$a_i > a_j$, а в перестановці $a_i$ знаходиться перед $a_j$. Кількість всіх пар елементів, що
утворюють інверсію, називають кількістю інверсій у даній перестановці.



Озн. Перестановка називається парною, якщо загальна кількість інверсій в ній
є парною і непарною --- в іншому випадку.



Приклад. Нехай $(1,3,5,2,4)$ --- це перестановка $5$-го порядку. Знайдемо
кількість інверсій: $S = 0 + 1 + 2 + 0 + 0 = 3$ інверсії, оскільки елемент $1$ утворює $0$
інверсій з рештою елементів перестановки; елемент $3$ --- одну інверсію з елементом
$2$, оскільки $3 > 2$; елемент $5$ утворює дві інверсії з елементами $2$ та $4$ $(5>2,5>4)$;
елемент $2$ і елемент $4$ інверсій не утворюють. Отже, дана перестановка є
непарною.



Озн. Підстановкою називається бієкція $f: A \rightarrow A$.

Іншими словами, підстановка --- це відображення однієї перестановки в іншу.
Щоб порахувати кількість інверсій підстановки, потрібно додати кількість інверсій
у кожній перестановці. Якщо отримана сума --- число парне, то підстановка
називається парною. Парність підстановки не залежить від її запису у вигляді двох
рядків. Тому прийнято перший рядок записувати у впорядкованому вигляді, тоді
кількість інверсій в цьому рядку буде дорівнювати нулю.


Приклад.

$f = \begin{pmatrix}
	1 & 3 & 5 & 2 & 4 \\
	4 & 1 & 3 & 5 & 2 \\
\end{pmatrix} = \begin{pmatrix}
	1 & 2 & 3 & 4 & 5 \\
	4 & 5 & 1 & 2 & 3 \\
\end{pmatrix}$, $s = 0 + (3 + 3) = 6$. Отже,
дана підстановка $f$ є парною.

Легко бачити, що формулу (1) можна переписати у вигляді

$$\varphi(x_1, x_2, x_3) = \varphi(e_1, e_2, e_3)\sum\limits_{\text{3-елементних перестановках}} (-1)^{s(i_1, i_2, i_3)} (a_{1 i_1}, a_{2 i_2}, a_{3 i_3}) (2)$$


де $s(i_1, i_2, i_3)$ --- це сума інверсій в перестановці $(i_1, i_2, i_3)$. 



